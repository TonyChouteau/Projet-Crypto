
% Default to the notebook output style

    


% Inherit from the specified cell style.




    
\documentclass{report}

    
    
    \usepackage[T1]{fontenc}
    % Nicer default font (+ math font) than Computer Modern for most use cases
    \usepackage{mathpazo}

    % Basic figure setup, for now with no caption control since it's done
    % automatically by Pandoc (which extracts ![](path) syntax from Markdown).
    \usepackage{graphicx}
    % We will generate all images so they have a width \maxwidth. This means
    % that they will get their normal width if they fit onto the page, but
    % are scaled down if they would overflow the margins.
    \makeatletter
    \def\maxwidth{\ifdim\Gin@nat@width>\linewidth\linewidth
    \else\Gin@nat@width\fi}
    \makeatother
    \let\Oldincludegraphics\includegraphics
    % Set max figure width to be 80% of text width, for now hardcoded.
    \renewcommand{\includegraphics}[1]{\Oldincludegraphics[width=.8\maxwidth]{#1}}
    % Ensure that by default, figures have no caption (until we provide a
    % proper Figure object with a Caption API and a way to capture that
    % in the conversion process - todo).
    \usepackage{caption}
    \DeclareCaptionLabelFormat{nolabel}{}
    \captionsetup{labelformat=nolabel}

    \usepackage{adjustbox} % Used to constrain images to a maximum size 
    \usepackage{xcolor} % Allow colors to be defined
    \usepackage{enumerate} % Needed for markdown enumerations to work
    \usepackage{geometry} % Used to adjust the document margins
    \usepackage{amsmath} % Equations
    \usepackage{amssymb} % Equations
    \usepackage{textcomp} % defines textquotesingle
    % Hack from http://tex.stackexchange.com/a/47451/13684:
    \AtBeginDocument{%
        \def\PYZsq{\textquotesingle}% Upright quotes in Pygmentized code
    }
    \usepackage{upquote} % Upright quotes for verbatim code
    \usepackage{eurosym} % defines \euro
    \usepackage[mathletters]{ucs} % Extended unicode (utf-8) support
    \usepackage[utf8x]{inputenc} % Allow utf-8 characters in the tex document
    \usepackage{fancyvrb} % verbatim replacement that allows latex
    \usepackage{grffile} % extends the file name processing of package graphics 
                         % to support a larger range 
    % The hyperref package gives us a pdf with properly built
    % internal navigation ('pdf bookmarks' for the table of contents,
    % internal cross-reference links, web links for URLs, etc.)
    \usepackage{hyperref}
    \usepackage{longtable} % longtable support required by pandoc >1.10
    \usepackage{booktabs}  % table support for pandoc > 1.12.2
    \usepackage[inline]{enumitem} % IRkernel/repr support (it uses the enumerate* environment)
    \usepackage[normalem]{ulem} % ulem is needed to support strikethroughs (\sout)
                                % normalem makes italics be italics, not underlines

    \usepackage{titlesec}
    \usepackage{framed,color}
    \definecolor{shadecolor}{rgb}{.92, .92, .92}
    
    \newenvironment{code}%
   {\snugshade}%
   {\endsnugshade}
   
 
    \titleformat{\chapter}[display]
      {\normalfont\bfseries}{}{0pt}{\huge}
      
      \titlespacing*{\chapter}{10pt}{10pt}{10pt}
      
     \titleformat{\subsubsection}{\filcenter}{}{}{}
     \titleformat{\paragraph}{\filcenter}{}{}{}
        

    \newcommand{\li}{\rule{\linewidth}{0.5pt}}
    
    % Colors for the hyperref package
    \definecolor{urlcolor}{rgb}{0,.145,.698}
    \definecolor{linkcolor}{rgb}{.71,0.21,0.01}
    \definecolor{citecolor}{rgb}{.12,.54,.11}

    % ANSI colors
    \definecolor{ansi-black}{HTML}{3E424D}
    \definecolor{ansi-black-intense}{HTML}{282C36}
    \definecolor{ansi-red}{HTML}{E75C58}
    \definecolor{ansi-red-intense}{HTML}{B22B31}
    \definecolor{ansi-green}{HTML}{00A250}
    \definecolor{ansi-green-intense}{HTML}{007427}
    \definecolor{ansi-yellow}{HTML}{DDB62B}
    \definecolor{ansi-yellow-intense}{HTML}{B27D12}
    \definecolor{ansi-blue}{HTML}{208FFB}
    \definecolor{ansi-blue-intense}{HTML}{0065CA}
    \definecolor{ansi-magenta}{HTML}{D160C4}
    \definecolor{ansi-magenta-intense}{HTML}{A03196}
    \definecolor{ansi-cyan}{HTML}{60C6C8}
    \definecolor{ansi-cyan-intense}{HTML}{258F8F}
    \definecolor{ansi-white}{HTML}{C5C1B4}
    \definecolor{ansi-white-intense}{HTML}{A1A6B2}
    
    \definecolor{verbatim}{gray}{0.93}

    % commands and environments needed by pandoc snippets
    % extracted from the output of `pandoc -s`
    \providecommand{\tightlist}{%
      \setlength{\itemsep}{0pt}\setlength{\parskip}{0pt}}
    \DefineVerbatimEnvironment{Highlighting}{Verbatim}{commandchars=\\\{\}}
    % Add ',fontsize=\small' for more characters per line
    \newenvironment{Shaded}{}{}
    \newcommand{\KeywordTok}[1]{\textcolor[rgb]{0.00,0.44,0.13}{\textbf{{#1}}}}
    \newcommand{\DataTypeTok}[1]{\textcolor[rgb]{0.56,0.13,0.00}{{#1}}}
    \newcommand{\DecValTok}[1]{\textcolor[rgb]{0.25,0.63,0.44}{{#1}}}
    \newcommand{\BaseNTok}[1]{\textcolor[rgb]{0.25,0.63,0.44}{{#1}}}
    \newcommand{\FloatTok}[1]{\textcolor[rgb]{0.25,0.63,0.44}{{#1}}}
    \newcommand{\CharTok}[1]{\textcolor[rgb]{0.25,0.44,0.63}{{#1}}}
    \newcommand{\StringTok}[1]{\textcolor[rgb]{0.25,0.44,0.63}{{#1}}}
    \newcommand{\CommentTok}[1]{\textcolor[rgb]{0.38,0.63,0.69}{\textit{{#1}}}}
    \newcommand{\OtherTok}[1]{\textcolor[rgb]{0.00,0.44,0.13}{{#1}}}
    \newcommand{\AlertTok}[1]{\textcolor[rgb]{1.00,0.00,0.00}{\textbf{{#1}}}}
    \newcommand{\FunctionTok}[1]{\textcolor[rgb]{0.02,0.16,0.49}{{#1}}}
    \newcommand{\RegionMarkerTok}[1]{{#1}}
    \newcommand{\ErrorTok}[1]{\textcolor[rgb]{1.00,0.00,0.00}{\textbf{{#1}}}}
    \newcommand{\NormalTok}[1]{{#1}}
    
    % Additional commands for more recent versions of Pandoc
    \newcommand{\ConstantTok}[1]{\textcolor[rgb]{0.53,0.00,0.00}{{#1}}}
    \newcommand{\SpecialCharTok}[1]{\textcolor[rgb]{0.25,0.44,0.63}{{#1}}}
    \newcommand{\VerbatimStringTok}[1]{\textcolor[rgb]{0.25,0.44,0.63}{{#1}}}
    \newcommand{\SpecialStringTok}[1]{\textcolor[rgb]{0.73,0.40,0.53}{{#1}}}
    \newcommand{\ImportTok}[1]{{#1}}
    \newcommand{\DocumentationTok}[1]{\textcolor[rgb]{0.73,0.13,0.13}{\textit{{#1}}}}
    \newcommand{\AnnotationTok}[1]{\textcolor[rgb]{0.38,0.63,0.69}{\textbf{\textit{{#1}}}}}
    \newcommand{\CommentVarTok}[1]{\textcolor[rgb]{0.38,0.63,0.69}{\textbf{\textit{{#1}}}}}
    \newcommand{\VariableTok}[1]{\textcolor[rgb]{0.10,0.09,0.49}{{#1}}}
    \newcommand{\ControlFlowTok}[1]{\textcolor[rgb]{0.00,0.44,0.13}{\textbf{{#1}}}}
    \newcommand{\OperatorTok}[1]{\textcolor[rgb]{0.40,0.40,0.40}{{#1}}}
    \newcommand{\BuiltInTok}[1]{{#1}}
    \newcommand{\ExtensionTok}[1]{{#1}}
    \newcommand{\PreprocessorTok}[1]{\textcolor[rgb]{0.74,0.48,0.00}{{#1}}}
    \newcommand{\AttributeTok}[1]{\textcolor[rgb]{0.49,0.56,0.16}{{#1}}}
    \newcommand{\InformationTok}[1]{\textcolor[rgb]{0.38,0.63,0.69}{\textbf{\textit{{#1}}}}}
    \newcommand{\WarningTok}[1]{\textcolor[rgb]{0.38,0.63,0.69}{\textbf{\textit{{#1}}}}}
    
    
    % Define a nice break command that doesn't care if a line doesn't already
    % exist.
    \def\br{\hspace*{\fill} \\* }
    % Math Jax compatability definitions
    \def\gt{>}
    \def\lt{<}
    % Document parameters
    \title{{\li\newline\huge \textbf{TP de Cryptologie}\par} }
    \author{\Large{Tony Chouteau - Info 2}}
    \date{\Large{Avril 2020}\\\newline\li\\\bigskip\normalsize ~ Fait via le Jupyter Notebook de Google Colab}

    % Pygments definitions
    
\makeatletter
\def\PY@reset{\let\PY@it=\relax \let\PY@bf=\relax%
    \let\PY@ul=\relax \let\PY@tc=\relax%
    \let\PY@bc=\relax \let\PY@ff=\relax}
\def\PY@tok#1{\csname PY@tok@#1\endcsname}
\def\PY@toks#1+{\ifx\relax#1\empty\else%
    \PY@tok{#1}\expandafter\PY@toks\fi}
\def\PY@do#1{\PY@bc{\PY@tc{\PY@ul{%
    \PY@it{\PY@bf{\PY@ff{#1}}}}}}}
\def\PY#1#2{\PY@reset\PY@toks#1+\relax+\PY@do{#2}}

\expandafter\def\csname PY@tok@w\endcsname{\def\PY@tc##1{\textcolor[rgb]{0.73,0.73,0.73}{##1}}}
\expandafter\def\csname PY@tok@c\endcsname{\let\PY@it=\textit\def\PY@tc##1{\textcolor[rgb]{0.25,0.50,0.50}{##1}}}
\expandafter\def\csname PY@tok@cp\endcsname{\def\PY@tc##1{\textcolor[rgb]{0.74,0.48,0.00}{##1}}}
\expandafter\def\csname PY@tok@k\endcsname{\let\PY@bf=\textbf\def\PY@tc##1{\textcolor[rgb]{0.00,0.50,0.00}{##1}}}
\expandafter\def\csname PY@tok@kp\endcsname{\def\PY@tc##1{\textcolor[rgb]{0.00,0.50,0.00}{##1}}}
\expandafter\def\csname PY@tok@kt\endcsname{\def\PY@tc##1{\textcolor[rgb]{0.69,0.00,0.25}{##1}}}
\expandafter\def\csname PY@tok@o\endcsname{\def\PY@tc##1{\textcolor[rgb]{0.40,0.40,0.40}{##1}}}
\expandafter\def\csname PY@tok@ow\endcsname{\let\PY@bf=\textbf\def\PY@tc##1{\textcolor[rgb]{0.67,0.13,1.00}{##1}}}
\expandafter\def\csname PY@tok@nb\endcsname{\def\PY@tc##1{\textcolor[rgb]{0.00,0.50,0.00}{##1}}}
\expandafter\def\csname PY@tok@nf\endcsname{\def\PY@tc##1{\textcolor[rgb]{0.00,0.00,1.00}{##1}}}
\expandafter\def\csname PY@tok@nc\endcsname{\let\PY@bf=\textbf\def\PY@tc##1{\textcolor[rgb]{0.00,0.00,1.00}{##1}}}
\expandafter\def\csname PY@tok@nn\endcsname{\let\PY@bf=\textbf\def\PY@tc##1{\textcolor[rgb]{0.00,0.00,1.00}{##1}}}
\expandafter\def\csname PY@tok@ne\endcsname{\let\PY@bf=\textbf\def\PY@tc##1{\textcolor[rgb]{0.82,0.25,0.23}{##1}}}
\expandafter\def\csname PY@tok@nv\endcsname{\def\PY@tc##1{\textcolor[rgb]{0.10,0.09,0.49}{##1}}}
\expandafter\def\csname PY@tok@no\endcsname{\def\PY@tc##1{\textcolor[rgb]{0.53,0.00,0.00}{##1}}}
\expandafter\def\csname PY@tok@nl\endcsname{\def\PY@tc##1{\textcolor[rgb]{0.63,0.63,0.00}{##1}}}
\expandafter\def\csname PY@tok@ni\endcsname{\let\PY@bf=\textbf\def\PY@tc##1{\textcolor[rgb]{0.60,0.60,0.60}{##1}}}
\expandafter\def\csname PY@tok@na\endcsname{\def\PY@tc##1{\textcolor[rgb]{0.49,0.56,0.16}{##1}}}
\expandafter\def\csname PY@tok@nt\endcsname{\let\PY@bf=\textbf\def\PY@tc##1{\textcolor[rgb]{0.00,0.50,0.00}{##1}}}
\expandafter\def\csname PY@tok@nd\endcsname{\def\PY@tc##1{\textcolor[rgb]{0.67,0.13,1.00}{##1}}}
\expandafter\def\csname PY@tok@s\endcsname{\def\PY@tc##1{\textcolor[rgb]{0.73,0.13,0.13}{##1}}}
\expandafter\def\csname PY@tok@sd\endcsname{\let\PY@it=\textit\def\PY@tc##1{\textcolor[rgb]{0.73,0.13,0.13}{##1}}}
\expandafter\def\csname PY@tok@si\endcsname{\let\PY@bf=\textbf\def\PY@tc##1{\textcolor[rgb]{0.73,0.40,0.53}{##1}}}
\expandafter\def\csname PY@tok@se\endcsname{\let\PY@bf=\textbf\def\PY@tc##1{\textcolor[rgb]{0.73,0.40,0.13}{##1}}}
\expandafter\def\csname PY@tok@sr\endcsname{\def\PY@tc##1{\textcolor[rgb]{0.73,0.40,0.53}{##1}}}
\expandafter\def\csname PY@tok@ss\endcsname{\def\PY@tc##1{\textcolor[rgb]{0.10,0.09,0.49}{##1}}}
\expandafter\def\csname PY@tok@sx\endcsname{\def\PY@tc##1{\textcolor[rgb]{0.00,0.50,0.00}{##1}}}
\expandafter\def\csname PY@tok@m\endcsname{\def\PY@tc##1{\textcolor[rgb]{0.40,0.40,0.40}{##1}}}
\expandafter\def\csname PY@tok@gh\endcsname{\let\PY@bf=\textbf\def\PY@tc##1{\textcolor[rgb]{0.00,0.00,0.50}{##1}}}
\expandafter\def\csname PY@tok@gu\endcsname{\let\PY@bf=\textbf\def\PY@tc##1{\textcolor[rgb]{0.50,0.00,0.50}{##1}}}
\expandafter\def\csname PY@tok@gd\endcsname{\def\PY@tc##1{\textcolor[rgb]{0.63,0.00,0.00}{##1}}}
\expandafter\def\csname PY@tok@gi\endcsname{\def\PY@tc##1{\textcolor[rgb]{0.00,0.63,0.00}{##1}}}
\expandafter\def\csname PY@tok@gr\endcsname{\def\PY@tc##1{\textcolor[rgb]{1.00,0.00,0.00}{##1}}}
\expandafter\def\csname PY@tok@ge\endcsname{\let\PY@it=\textit}
\expandafter\def\csname PY@tok@gs\endcsname{\let\PY@bf=\textbf}
\expandafter\def\csname PY@tok@gp\endcsname{\let\PY@bf=\textbf\def\PY@tc##1{\textcolor[rgb]{0.00,0.00,0.50}{##1}}}
\expandafter\def\csname PY@tok@go\endcsname{\def\PY@tc##1{\textcolor[rgb]{0.53,0.53,0.53}{##1}}}
\expandafter\def\csname PY@tok@gt\endcsname{\def\PY@tc##1{\textcolor[rgb]{0.00,0.27,0.87}{##1}}}
\expandafter\def\csname PY@tok@err\endcsname{\def\PY@bc##1{\setlength{\fboxsep}{0pt}\fcolorbox[rgb]{1.00,0.00,0.00}{1,1,1}{\strut ##1}}}
\expandafter\def\csname PY@tok@kc\endcsname{\let\PY@bf=\textbf\def\PY@tc##1{\textcolor[rgb]{0.00,0.50,0.00}{##1}}}
\expandafter\def\csname PY@tok@kd\endcsname{\let\PY@bf=\textbf\def\PY@tc##1{\textcolor[rgb]{0.00,0.50,0.00}{##1}}}
\expandafter\def\csname PY@tok@kn\endcsname{\let\PY@bf=\textbf\def\PY@tc##1{\textcolor[rgb]{0.00,0.50,0.00}{##1}}}
\expandafter\def\csname PY@tok@kr\endcsname{\let\PY@bf=\textbf\def\PY@tc##1{\textcolor[rgb]{0.00,0.50,0.00}{##1}}}
\expandafter\def\csname PY@tok@bp\endcsname{\def\PY@tc##1{\textcolor[rgb]{0.00,0.50,0.00}{##1}}}
\expandafter\def\csname PY@tok@fm\endcsname{\def\PY@tc##1{\textcolor[rgb]{0.00,0.00,1.00}{##1}}}
\expandafter\def\csname PY@tok@vc\endcsname{\def\PY@tc##1{\textcolor[rgb]{0.10,0.09,0.49}{##1}}}
\expandafter\def\csname PY@tok@vg\endcsname{\def\PY@tc##1{\textcolor[rgb]{0.10,0.09,0.49}{##1}}}
\expandafter\def\csname PY@tok@vi\endcsname{\def\PY@tc##1{\textcolor[rgb]{0.10,0.09,0.49}{##1}}}
\expandafter\def\csname PY@tok@vm\endcsname{\def\PY@tc##1{\textcolor[rgb]{0.10,0.09,0.49}{##1}}}
\expandafter\def\csname PY@tok@sa\endcsname{\def\PY@tc##1{\textcolor[rgb]{0.73,0.13,0.13}{##1}}}
\expandafter\def\csname PY@tok@sb\endcsname{\def\PY@tc##1{\textcolor[rgb]{0.73,0.13,0.13}{##1}}}
\expandafter\def\csname PY@tok@sc\endcsname{\def\PY@tc##1{\textcolor[rgb]{0.73,0.13,0.13}{##1}}}
\expandafter\def\csname PY@tok@dl\endcsname{\def\PY@tc##1{\textcolor[rgb]{0.73,0.13,0.13}{##1}}}
\expandafter\def\csname PY@tok@s2\endcsname{\def\PY@tc##1{\textcolor[rgb]{0.73,0.13,0.13}{##1}}}
\expandafter\def\csname PY@tok@sh\endcsname{\def\PY@tc##1{\textcolor[rgb]{0.73,0.13,0.13}{##1}}}
\expandafter\def\csname PY@tok@s1\endcsname{\def\PY@tc##1{\textcolor[rgb]{0.73,0.13,0.13}{##1}}}
\expandafter\def\csname PY@tok@mb\endcsname{\def\PY@tc##1{\textcolor[rgb]{0.40,0.40,0.40}{##1}}}
\expandafter\def\csname PY@tok@mf\endcsname{\def\PY@tc##1{\textcolor[rgb]{0.40,0.40,0.40}{##1}}}
\expandafter\def\csname PY@tok@mh\endcsname{\def\PY@tc##1{\textcolor[rgb]{0.40,0.40,0.40}{##1}}}
\expandafter\def\csname PY@tok@mi\endcsname{\def\PY@tc##1{\textcolor[rgb]{0.40,0.40,0.40}{##1}}}
\expandafter\def\csname PY@tok@il\endcsname{\def\PY@tc##1{\textcolor[rgb]{0.40,0.40,0.40}{##1}}}
\expandafter\def\csname PY@tok@mo\endcsname{\def\PY@tc##1{\textcolor[rgb]{0.40,0.40,0.40}{##1}}}
\expandafter\def\csname PY@tok@ch\endcsname{\let\PY@it=\textit\def\PY@tc##1{\textcolor[rgb]{0.25,0.50,0.50}{##1}}}
\expandafter\def\csname PY@tok@cm\endcsname{\let\PY@it=\textit\def\PY@tc##1{\textcolor[rgb]{0.25,0.50,0.50}{##1}}}
\expandafter\def\csname PY@tok@cpf\endcsname{\let\PY@it=\textit\def\PY@tc##1{\textcolor[rgb]{0.25,0.50,0.50}{##1}}}
\expandafter\def\csname PY@tok@c1\endcsname{\let\PY@it=\textit\def\PY@tc##1{\textcolor[rgb]{0.25,0.50,0.50}{##1}}}
\expandafter\def\csname PY@tok@cs\endcsname{\let\PY@it=\textit\def\PY@tc##1{\textcolor[rgb]{0.25,0.50,0.50}{##1}}}

\def\PYZbs{\char`\\}
\def\PYZus{\char`\_}
\def\PYZob{\char`\{}
\def\PYZcb{\char`\}}
\def\PYZca{\char`\^}
\def\PYZam{\char`\&}
\def\PYZlt{\char`\<}
\def\PYZgt{\char`\>}
\def\PYZsh{\char`\#}
\def\PYZpc{\char`\%}
\def\PYZdl{\char`\$}
\def\PYZhy{\char`\-}
\def\PYZsq{\char`\'}
\def\PYZdq{\char`\"}
\def\PYZti{\char`\~}
% for compatibility with earlier versions
\def\PYZat{@}
\def\PYZlb{[}
\def\PYZrb{]}



\makeatother


    % Exact colors from NB
    \definecolor{incolor}{rgb}{0.0, 0.0, 0.5}
    \definecolor{outcolor}{rgb}{0.545, 0.0, 0.0}



    
    % Prevent overflowing lines due to hard-to-break entities
    \sloppy 
    % Setup hyperref package
    \hypersetup{
      breaklinks=true,  % so long urls are correctly broken across lines
      colorlinks=true,
      urlcolor=urlcolor,
      linkcolor=linkcolor,
      citecolor=citecolor,
      }
    % Slightly bigger margins than the latex defaults
    
    \geometry{verbose,tmargin=1.5cm,bmargin=1.5cm,lmargin=1.5cm,rmargin=1.5cm}
    
    

    \begin{document}
    
    
    \maketitle
    
    
    \tableofcontents

\begin{center}\rule{0.5\linewidth}{\linethickness}\end{center}


\chapter{\texorpdfstring{\textbf{Introduction}}{Introduction}}\label{introduction}

\subsubsection{\texorpdfstring{\textbf{Imports}}{Imports}}\label{imports}

    \begin{code}\begin{Verbatim}[commandchars=\\\{\}]
{\color{incolor}In [{\color{incolor}0}]:} \PY{k+kn}{import} \PY{n+nn}{random} \PY{k}{as} \PY{n+nn}{rand}
        \PY{k+kn}{import} \PY{n+nn}{sys}
        \PY{k+kn}{import} \PY{n+nn}{sympy} \PY{k}{as} \PY{n+nn}{sy}
        \PY{k+kn}{import} \PY{n+nn}{time}
        \PY{k+kn}{import} \PY{n+nn}{copy}
        \PY{c+c1}{\PYZsh{}import math}
\end{Verbatim}
\end{code}

\subsubsection{\texorpdfstring{\textbf{Définition}}{Définition}}\label{duxe9finition}

J'utilise pour ce projet:
\begin{enumerate}
    \item 
La bibliothèque \(sys\) pour augmenter le nombre maximum de
récursion (utile pour la génération de très grands nombres premiers).
\item
La bibliothèque \(sympy\) pour l'ensemble de ses fonctions
permettant de trouver, vérifier des nombres premiers, et donc tester mes
algorithmes.
\item
La bibliothèque \(time\) pour mesurer la complexité temporelle
réelle de mess algorithmes. J'en profite pour définir une fonction
\(millis()\) qui renvoie le nombre de millisecondes depuis \emph{epoch}, soit
le \emph{1er janvier 1970, 00:00:00 (UTC)}.

\end{enumerate}

    \begin{code}\begin{Verbatim}[commandchars=\\\{\}]
{\color{incolor}In [{\color{incolor}183}]:} \PY{n+nb}{print}\PY{p}{(}\PY{l+s+s2}{\PYZdq{}}\PY{l+s+s2}{La version de python utilisé est :}\PY{l+s+s2}{\PYZdq{}}\PY{p}{,} \PY{n}{sys}\PY{o}{.}\PY{n}{version}\PY{p}{)}
          \PY{n+nb}{print}\PY{p}{(}\PY{l+s+s2}{\PYZdq{}}\PY{l+s+s2}{\PYZdq{}}\PY{p}{)}
          
          \PY{n}{limit} \PY{o}{=} \PY{l+m+mi}{10}\PY{o}{*}\PY{o}{*}\PY{l+m+mi}{7}
          \PY{n+nb}{print}\PY{p}{(}\PY{l+s+s2}{\PYZdq{}}\PY{l+s+s2}{Nombre de récursions maximum définie à :}\PY{l+s+s2}{\PYZdq{}}\PY{p}{,}\PY{n}{limit}\PY{p}{)}
          \PY{n}{sys}\PY{o}{.}\PY{n}{setrecursionlimit}\PY{p}{(}\PY{n}{limit}\PY{p}{)}
          \PY{n+nb}{print}\PY{p}{(}\PY{l+s+s2}{\PYZdq{}}\PY{l+s+s2}{\PYZdq{}}\PY{p}{)}
          
          \PY{n}{millis} \PY{o}{=} \PY{k}{lambda}\PY{p}{:} \PY{n+nb}{int}\PY{p}{(}\PY{n+nb}{round}\PY{p}{(}\PY{n}{time}\PY{o}{.}\PY{n}{time}\PY{p}{(}\PY{p}{)} \PY{o}{*} \PY{l+m+mi}{10}\PY{o}{*}\PY{o}{*}\PY{l+m+mi}{3}\PY{p}{)}\PY{p}{)}
          \PY{n+nb}{print}\PY{p}{(}\PY{l+s+s2}{\PYZdq{}}\PY{l+s+s2}{Milliseconde actuelle :}\PY{l+s+s2}{\PYZdq{}}\PY{p}{,}\PY{n}{millis}\PY{p}{(}\PY{p}{)}\PY{p}{)}
          
          \PY{n}{micro} \PY{o}{=} \PY{k}{lambda}\PY{p}{:} \PY{n+nb}{int}\PY{p}{(}\PY{n+nb}{round}\PY{p}{(}\PY{n}{time}\PY{o}{.}\PY{n}{time}\PY{p}{(}\PY{p}{)} \PY{o}{*} \PY{l+m+mi}{10}\PY{o}{*}\PY{o}{*}\PY{l+m+mi}{6}\PY{p}{)}\PY{p}{)}
          \PY{n+nb}{print}\PY{p}{(}\PY{l+s+s2}{\PYZdq{}}\PY{l+s+s2}{MicroSeconde actuelle :}\PY{l+s+s2}{\PYZdq{}}\PY{p}{,}\PY{n}{micro}\PY{p}{(}\PY{p}{)}\PY{p}{)}
          
          \PY{k}{def} \PY{n+nf}{displayPeriod}\PY{p}{(}\PY{n}{t}\PY{p}{)}\PY{p}{:}
            \PY{k}{if} \PY{p}{(}\PY{n}{t}\PY{o}{\PYZlt{}}\PY{l+m+mi}{10}\PY{o}{*}\PY{o}{*}\PY{l+m+mi}{3}\PY{p}{)}\PY{p}{:}
              \PY{k}{return} \PY{p}{(}\PY{n+nb}{str}\PY{p}{(}\PY{n}{t}\PY{p}{)}\PY{o}{+}\PY{l+s+s2}{\PYZdq{}}\PY{l+s+s2}{µs}\PY{l+s+s2}{\PYZdq{}}\PY{p}{)}
            \PY{k}{if} \PY{p}{(}\PY{n}{t}\PY{o}{\PYZlt{}}\PY{l+m+mi}{10}\PY{o}{*}\PY{o}{*}\PY{l+m+mi}{6}\PY{p}{)}\PY{p}{:}
              \PY{k}{return} \PY{p}{(}\PY{n+nb}{str}\PY{p}{(}\PY{n}{t}\PY{o}{/}\PY{l+m+mi}{10}\PY{o}{*}\PY{o}{*}\PY{l+m+mi}{3}\PY{p}{)}\PY{o}{+}\PY{l+s+s2}{\PYZdq{}}\PY{l+s+s2}{ms}\PY{l+s+s2}{\PYZdq{}}\PY{p}{)}
            \PY{k}{else}\PY{p}{:}
              \PY{k}{return} \PY{p}{(}\PY{n+nb}{str}\PY{p}{(}\PY{n}{t}\PY{o}{/}\PY{l+m+mi}{10}\PY{o}{*}\PY{o}{*}\PY{l+m+mi}{6}\PY{p}{)}\PY{o}{+}\PY{l+s+s2}{\PYZdq{}}\PY{l+s+s2}{s}\PY{l+s+s2}{\PYZdq{}}\PY{p}{)}
\end{Verbatim}
\end{code}


    \begin{code}\begin{Verbatim}[commandchars=\\\{\}]
La version de python utilisé est : 3.6.9 (default, Nov  7 2019, 10:44:02) 
[GCC 8.3.0]

Nombre de récursions maximum définie à : 10000000

Milliseconde actuelle : 1587312760032
MicroSeconde actuelle : 1587312760032500

    \end{Verbatim}
    \end{code}

\chapter{\texorpdfstring{\textbf{Partie 1 - Arithmétique dans Z et Z/nZ
avec
Python}}{Partie 1 - Arithmétique dans Z et Z/nZ avec Python}}\label{partie-1---arithmuxe9tique-dans-z-et-znz-avec-python}

\section{\texorpdfstring{\textbf{Question
1}}{Question 1}}\label{question-1}

\subsection{\texorpdfstring{\textbf{Euclide
Simple}}{Euclide Simple}}\label{euclide-simple}

\subsubsection{\texorpdfstring{\textbf{Définition}}{Définition}}\label{duxe9finition}


    \begin{code}\begin{Verbatim}[commandchars=\\\{\}]
{\color{incolor}In [{\color{incolor}0}]:} \PY{k}{def} \PY{n+nf}{euclideSimple}\PY{p}{(}\PY{n}{a}\PY{p}{,} \PY{n}{b}\PY{p}{)}\PY{p}{:}
          \PY{k}{assert} \PY{p}{(}\PY{n}{a} \PY{o}{\PYZgt{}} \PY{l+m+mi}{0} \PY{o+ow}{and} \PY{n}{b} \PY{o}{\PYZgt{}} \PY{l+m+mi}{0}\PY{p}{)}\PY{p}{,} \PY{l+s+s2}{\PYZdq{}}\PY{l+s+s2}{a et b doivent être positifs}\PY{l+s+s2}{\PYZdq{}}
          
          \PY{k}{if} \PY{n}{a} \PY{o}{\PYZlt{}} \PY{n}{b}\PY{p}{:}
            \PY{n}{a}\PY{p}{,} \PY{n}{b} \PY{o}{=} \PY{n}{b}\PY{p}{,} \PY{n}{a}
        
          \PY{n}{a0}\PY{p}{,} \PY{n}{b0} \PY{o}{=} \PY{n}{a}\PY{p}{,} \PY{n}{b}
          
          \PY{k}{while} \PY{n}{b}\PY{p}{:}
            \PY{n}{a}\PY{p}{,} \PY{n}{b} \PY{o}{=} \PY{n}{b}\PY{p}{,} \PY{n}{a} \PY{o}{\PYZpc{}} \PY{n}{b}
          
          \PY{k}{return} \PY{n}{a}\PY{p}{,} \PY{n}{a0}\PY{o}{/}\PY{o}{/}\PY{n}{a}\PY{p}{,} \PY{n}{b0}\PY{o}{/}\PY{o}{/}\PY{n}{a}
\end{Verbatim}
\end{code}

\subsubsection{\texorpdfstring{\textbf{Test}}{Test}}\label{test}

    \begin{code}\begin{Verbatim}[commandchars=\\\{\}]
{\color{incolor}In [{\color{incolor}185}]:} \PY{n}{a} \PY{o}{=} \PY{l+m+mi}{60}
          \PY{n}{b} \PY{o}{=} \PY{l+m+mi}{45}
          
          \PY{n}{r}\PY{p}{,} \PY{n}{a0}\PY{p}{,} \PY{n}{b0} \PY{o}{=} \PY{n}{euclideSimple}\PY{p}{(}\PY{n}{a}\PY{p}{,} \PY{n}{b}\PY{p}{)}
          
          \PY{n+nb}{print}\PY{p}{(}\PY{l+s+s2}{\PYZdq{}}\PY{l+s+s2}{Le PGCD de}\PY{l+s+s2}{\PYZdq{}}\PY{p}{,} \PY{n}{a}\PY{p}{,} \PY{l+s+s2}{\PYZdq{}}\PY{l+s+s2}{et de}\PY{l+s+s2}{\PYZdq{}}\PY{p}{,} \PY{n}{b}\PY{p}{,} \PY{l+s+s2}{\PYZdq{}}\PY{l+s+s2}{est :}\PY{l+s+s2}{\PYZdq{}}\PY{p}{,}\PY{n}{r}\PY{p}{)}
          \PY{n+nb}{print}\PY{p}{(}\PY{l+s+s2}{\PYZdq{}}\PY{l+s+s2}{\PYZdq{}}\PY{p}{)}
          \PY{n+nb}{print}\PY{p}{(}\PY{l+s+s2}{\PYZdq{}}\PY{l+s+s2}{Ainsi : }\PY{l+s+s2}{\PYZdq{}}\PY{p}{)}
          \PY{n+nb}{print}\PY{p}{(}\PY{n}{a0}\PY{p}{,} \PY{l+s+s2}{\PYZdq{}}\PY{l+s+s2}{*}\PY{l+s+s2}{\PYZdq{}}\PY{p}{,} \PY{n}{r}\PY{p}{,} \PY{l+s+s2}{\PYZdq{}}\PY{l+s+s2}{=}\PY{l+s+s2}{\PYZdq{}}\PY{p}{,} \PY{n}{a}\PY{p}{)}
          \PY{n+nb}{print}\PY{p}{(}\PY{n}{b0}\PY{p}{,} \PY{l+s+s2}{\PYZdq{}}\PY{l+s+s2}{*}\PY{l+s+s2}{\PYZdq{}}\PY{p}{,} \PY{n}{r}\PY{p}{,} \PY{l+s+s2}{\PYZdq{}}\PY{l+s+s2}{=}\PY{l+s+s2}{\PYZdq{}}\PY{p}{,} \PY{n}{b}\PY{p}{)}
\end{Verbatim}
\end{code}

    \begin{code}\begin{Verbatim}[commandchars=\\\{\}]
Le PGCD de 60 et de 45 est : 15

Ainsi : 
4 * 15 = 60
3 * 15 = 45

    \end{Verbatim}
    \end{code}

\subsection{\texorpdfstring{\textbf{Euclide
Étendu}}{Euclide Étendu}}\label{euclide-uxe9tendu}

\subsubsection{\texorpdfstring{\textbf{Définition}}{Définition}}\label{duxe9finition}

    \begin{code}\begin{Verbatim}[commandchars=\\\{\}]
{\color{incolor}In [{\color{incolor}0}]:} \PY{k}{def} \PY{n+nf}{euclideEtendu}\PY{p}{(}\PY{n}{a}\PY{p}{,} \PY{n}{b}\PY{p}{)}\PY{p}{:}
          \PY{k}{assert} \PY{n+nb}{type}\PY{p}{(}\PY{n}{a}\PY{p}{)}\PY{o}{==}\PY{n+nb}{type}\PY{p}{(}\PY{n}{b}\PY{p}{)}\PY{p}{,} \PY{l+s+s2}{\PYZdq{}}\PY{l+s+s2}{Les deux paramêtres doivent avoir le même type}\PY{l+s+s2}{\PYZdq{}}
          
          \PY{n}{r}\PY{p}{,} \PY{n}{u}\PY{p}{,} \PY{n}{v}\PY{p}{,} \PY{n}{r0}\PY{p}{,} \PY{n}{u0}\PY{p}{,} \PY{n}{v0} \PY{o}{=} \PY{n}{a}\PY{p}{,} \PY{l+m+mi}{1}\PY{p}{,} \PY{l+m+mi}{0}\PY{p}{,} \PY{n}{b}\PY{p}{,} \PY{l+m+mi}{0}\PY{p}{,} \PY{l+m+mi}{1}
          
          \PY{k}{while} \PY{n}{r0}\PY{p}{:}
            \PY{n}{q} \PY{o}{=} \PY{n}{r}\PY{o}{/}\PY{o}{/}\PY{n}{r0}
            \PY{n}{r}\PY{p}{,} \PY{n}{u}\PY{p}{,} \PY{n}{v}\PY{p}{,} \PY{n}{r0}\PY{p}{,} \PY{n}{u0}\PY{p}{,} \PY{n}{v0} \PY{o}{=} \PY{n}{r0}\PY{p}{,} \PY{n}{u0}\PY{p}{,} \PY{n}{v0}\PY{p}{,} \PY{n}{r}\PY{o}{\PYZhy{}}\PY{n}{q}\PY{o}{*}\PY{n}{r0}\PY{p}{,} \PY{n}{u}\PY{o}{\PYZhy{}}\PY{n}{q}\PY{o}{*}\PY{n}{u0}\PY{p}{,} \PY{n}{v}\PY{o}{\PYZhy{}}\PY{n}{q}\PY{o}{*}\PY{n}{v0}
        
          \PY{k}{return} \PY{n}{r}\PY{p}{,} \PY{n}{u}\PY{p}{,} \PY{n}{v}
\end{Verbatim}
\end{code}

\subsubsection{\texorpdfstring{\textbf{Test}}{Test}}\label{test}

    \begin{code}\begin{Verbatim}[commandchars=\\\{\}]
{\color{incolor}In [{\color{incolor}187}]:} \PY{n}{a} \PY{o}{=} \PY{l+m+mi}{45}
          \PY{n}{b} \PY{o}{=} \PY{l+m+mi}{60}
          
          \PY{n}{r}\PY{p}{,} \PY{n}{u}\PY{p}{,} \PY{n}{v} \PY{o}{=} \PY{n}{euclideEtendu}\PY{p}{(}\PY{n}{a}\PY{p}{,} \PY{n}{b}\PY{p}{)}
          
          \PY{n+nb}{print}\PY{p}{(}\PY{l+s+s2}{\PYZdq{}}\PY{l+s+s2}{Le PGCD de}\PY{l+s+s2}{\PYZdq{}}\PY{p}{,} \PY{n}{a}\PY{p}{,} \PY{l+s+s2}{\PYZdq{}}\PY{l+s+s2}{et de}\PY{l+s+s2}{\PYZdq{}}\PY{p}{,} \PY{n}{b}\PY{p}{,} \PY{l+s+s2}{\PYZdq{}}\PY{l+s+s2}{est :}\PY{l+s+s2}{\PYZdq{}}\PY{p}{,}\PY{n}{r}\PY{p}{)}
          \PY{n+nb}{print}\PY{p}{(}\PY{l+s+s2}{\PYZdq{}}\PY{l+s+s2}{\PYZdq{}}\PY{p}{)}
          \PY{n+nb}{print}\PY{p}{(}\PY{l+s+s2}{\PYZdq{}}\PY{l+s+s2}{De plus : r=au+bv avec u =}\PY{l+s+s2}{\PYZdq{}}\PY{p}{,}\PY{n}{u}\PY{p}{,}\PY{l+s+s2}{\PYZdq{}}\PY{l+s+s2}{ et v =}\PY{l+s+s2}{\PYZdq{}}\PY{p}{,}\PY{n}{v}\PY{p}{)}
          \PY{n+nb}{print}\PY{p}{(}\PY{l+s+s2}{\PYZdq{}}\PY{l+s+s2}{\PYZdq{}}\PY{p}{)}
          \PY{n+nb}{print}\PY{p}{(}\PY{n}{r}\PY{p}{,} \PY{l+s+s2}{\PYZdq{}}\PY{l+s+s2}{=}\PY{l+s+s2}{\PYZdq{}}\PY{p}{,} \PY{n}{a}\PY{p}{,} \PY{l+s+s2}{\PYZdq{}}\PY{l+s+s2}{*}\PY{l+s+s2}{\PYZdq{}}\PY{p}{,} \PY{n}{u}\PY{p}{,} \PY{l+s+s2}{\PYZdq{}}\PY{l+s+s2}{+}\PY{l+s+s2}{\PYZdq{}}\PY{p}{,} \PY{n}{b}\PY{p}{,} \PY{l+s+s2}{\PYZdq{}}\PY{l+s+s2}{*}\PY{l+s+s2}{\PYZdq{}}\PY{p}{,} \PY{n}{v}\PY{p}{,} \PY{l+s+s2}{\PYZdq{}}\PY{l+s+s2}{=}\PY{l+s+s2}{\PYZdq{}}\PY{p}{,} \PY{n}{a}\PY{o}{*}\PY{n}{u}\PY{p}{,} \PY{l+s+s2}{\PYZdq{}}\PY{l+s+s2}{+}\PY{l+s+s2}{\PYZdq{}}\PY{p}{,} \PY{n}{b}\PY{o}{*}\PY{n}{v}\PY{p}{)}
\end{Verbatim}
\end{code}

    \begin{code}\begin{Verbatim}[commandchars=\\\{\}]
Le PGCD de 45 et de 60 est : 15

De plus : r=au+bv avec u = -1  et v = 1

15 = 45 * -1 + 60 * 1 = -45 + 60

    \end{Verbatim}
    \end{code}

\section{\texorpdfstring{\textbf{Question
2}}{Question 2}}\label{question-2}

\subsection{\texorpdfstring{\textbf{Inverse
Modulaire}}{Inverse Modulaire}}\label{inverse-modulaire}

\subsubsection{\texorpdfstring{\textbf{Définition}}{Définition}}\label{duxe9finition}

    \begin{code}\begin{Verbatim}[commandchars=\\\{\}]
{\color{incolor}In [{\color{incolor}0}]:} \PY{k}{def} \PY{n+nf}{inverseMod}\PY{p}{(}\PY{n}{x}\PY{p}{,} \PY{n}{n}\PY{p}{)}\PY{p}{:}
          \PY{n}{r}\PY{p}{,} \PY{n}{u}\PY{p}{,} \PY{n}{\PYZus{}} \PY{o}{=} \PY{n}{euclideEtendu}\PY{p}{(}\PY{n}{x}\PY{p}{,} \PY{n}{n}\PY{p}{)}
          
          \PY{k}{assert} \PY{n}{r}\PY{o}{==}\PY{l+m+mi}{1}\PY{p}{,} \PY{l+s+s2}{\PYZdq{}}\PY{l+s+s2}{x et n doivent être premiers entre eux}\PY{l+s+s2}{\PYZdq{}} 
        
          \PY{k}{return} \PY{n}{u}\PY{o}{\PYZpc{}}\PY{k}{n}
\end{Verbatim}
\end{code}


\subsubsection{\texorpdfstring{\textbf{Test}}{Test}}\label{test}

   \begin{code}\begin{Verbatim}[commandchars=\\\{\}]
{\color{incolor}In [{\color{incolor}189}]:} \PY{n}{a} \PY{o}{=} \PY{l+m+mi}{23}
          \PY{n}{n} \PY{o}{=} \PY{l+m+mi}{29}
          
          \PY{n}{inv} \PY{o}{=} \PY{n}{inverseMod}\PY{p}{(}\PY{n}{a}\PY{p}{,} \PY{n}{n}\PY{p}{)}
          
          \PY{n+nb}{print}\PY{p}{(}\PY{l+s+s2}{\PYZdq{}}\PY{l+s+s2}{L}\PY{l+s+s2}{\PYZsq{}}\PY{l+s+s2}{inverse de}\PY{l+s+s2}{\PYZdq{}}\PY{p}{,}\PY{n}{a}\PY{p}{,}\PY{l+s+s2}{\PYZdq{}}\PY{l+s+s2}{modulo}\PY{l+s+s2}{\PYZdq{}}\PY{p}{,}\PY{n}{n}\PY{p}{,}\PY{l+s+s2}{\PYZdq{}}\PY{l+s+s2}{est :}\PY{l+s+s2}{\PYZdq{}}\PY{p}{,}\PY{n}{inv}\PY{p}{)}
\end{Verbatim}
\end{code}


   \begin{code}\begin{Verbatim}[commandchars=\\\{\}]
L'inverse de 23 modulo 29 est : 24

    \end{Verbatim}
\end{code}

\section{\texorpdfstring{\textbf{Question
3}}{Question 3}}\label{question-3}

\subsection{\texorpdfstring{\textbf{Exponentiation
Rapide}}{Exponentiation Rapide}}\label{exponentiation-rapide}

\subsubsection{\texorpdfstring{\textbf{Définition}}{Définition}}\label{duxe9finition}

   \begin{code}\begin{Verbatim}[commandchars=\\\{\}]
{\color{incolor}In [{\color{incolor}0}]:} \PY{k}{def} \PY{n+nf}{expoRapide}\PY{p}{(}\PY{n}{x}\PY{p}{,} \PY{n}{k}\PY{p}{,} \PY{n}{n}\PY{o}{=}\PY{k+kc}{None}\PY{p}{)}\PY{p}{:}
          \PY{k}{if} \PY{p}{(}\PY{n}{k} \PY{o}{==} \PY{o}{\PYZhy{}}\PY{l+m+mi}{1} \PY{o+ow}{and} \PY{n}{n} \PY{o}{!=} \PY{k+kc}{None}\PY{p}{)}\PY{p}{:} \PY{c+c1}{\PYZsh{}Inverse modulaire}
            \PY{k}{return} \PY{n}{inverseMod}\PY{p}{(}\PY{n}{x}\PY{p}{,}\PY{n}{n}\PY{p}{)}
          \PY{k}{if} \PY{p}{(}\PY{n}{k} \PY{o}{==} \PY{l+m+mi}{1}\PY{p}{)}\PY{p}{:}
            \PY{k}{if} \PY{p}{(}\PY{n}{n}\PY{o}{!=}\PY{k+kc}{None}\PY{p}{)}\PY{p}{:}
              \PY{k}{return} \PY{n}{x}\PY{o}{\PYZpc{}}\PY{k}{n}
            \PY{k}{else}\PY{p}{:}
              \PY{k}{return} \PY{n}{x}
          \PY{k}{if} \PY{p}{(}\PY{o+ow}{not} \PY{n}{k}\PY{o}{\PYZpc{}}\PY{k}{2}): \PY{c+c1}{\PYZsh{}Paire}
            \PY{k}{if} \PY{p}{(}\PY{n}{n}\PY{o}{!=}\PY{k+kc}{None}\PY{p}{)}\PY{p}{:}
              \PY{k}{return} \PY{n}{expoRapide}\PY{p}{(}\PY{n}{x}\PY{o}{*}\PY{n}{x}\PY{o}{\PYZpc{}}\PY{k}{n}, k//2, n)\PYZpc{}n
            \PY{k}{else}\PY{p}{:}
              \PY{k}{return} \PY{n}{expoRapide}\PY{p}{(}\PY{n}{x}\PY{o}{*}\PY{n}{x}\PY{p}{,} \PY{n}{k}\PY{o}{/}\PY{o}{/}\PY{l+m+mi}{2}\PY{p}{,} \PY{n}{n}\PY{p}{)}
          \PY{k}{if} \PY{p}{(}\PY{n}{k}\PY{o}{\PYZpc{}}\PY{k}{2}): \PY{c+c1}{\PYZsh{}Impaire}
            \PY{k}{if} \PY{p}{(}\PY{n}{n}\PY{o}{!=}\PY{k+kc}{None}\PY{p}{)}\PY{p}{:}
              \PY{k}{return} \PY{n}{x}\PY{o}{*}\PY{n}{expoRapide}\PY{p}{(}\PY{n}{x}\PY{o}{*}\PY{n}{x}\PY{o}{\PYZpc{}}\PY{k}{n}, (k\PYZhy{}1)//2, n)\PYZpc{}n
            \PY{k}{else}\PY{p}{:}
              \PY{k}{return} \PY{n}{x}\PY{o}{*}\PY{n}{expoRapide}\PY{p}{(}\PY{n}{x}\PY{o}{*}\PY{n}{x}\PY{p}{,} \PY{p}{(}\PY{n}{k}\PY{o}{\PYZhy{}}\PY{l+m+mi}{1}\PY{p}{)}\PY{o}{/}\PY{o}{/}\PY{l+m+mi}{2}\PY{p}{,} \PY{n}{n}\PY{p}{)}
\end{Verbatim}
\end{code}


\subsubsection{\texorpdfstring{\textbf{Test}}{Test}}\label{test}

   \begin{code}\begin{Verbatim}[commandchars=\\\{\}]
{\color{incolor}In [{\color{incolor}191}]:} \PY{n}{x} \PY{o}{=} \PY{l+m+mi}{4}
          \PY{n}{k} \PY{o}{=} \PY{l+m+mi}{87}
          
          \PY{n}{r} \PY{o}{=} \PY{n}{expoRapide}\PY{p}{(}\PY{n}{x}\PY{p}{,} \PY{n}{k}\PY{p}{)}
          
          \PY{n+nb}{print}\PY{p}{(}\PY{n}{x}\PY{p}{,}\PY{l+s+s2}{\PYZdq{}}\PY{l+s+s2}{à la puissance}\PY{l+s+s2}{\PYZdq{}}\PY{p}{,}\PY{n}{k}\PY{p}{,}\PY{l+s+s2}{\PYZdq{}}\PY{l+s+s2}{donne :}\PY{l+s+s2}{\PYZdq{}}\PY{p}{,}\PY{n}{r}\PY{p}{)}
\end{Verbatim}
\end{code}


   \begin{code}\begin{Verbatim}[commandchars=\\\{\}]
4 à la puissance 87 donne : 23945242826029513411849172299223580994042798784118784

    \end{Verbatim}
\end{code}

\section{\texorpdfstring{\textbf{Question
4}}{Question 4}}\label{question-4}

\subsection{\texorpdfstring{\textbf{IntegerModRing}}{IntegerModRing}}\label{integermodring}

Utilisant Jupyter je n'ai pas accès à l'objet \(IntegerModRing\) de
SAGE, je ne peux donc pas proposer d'études temporelle.
Je l'ai tout de même re-créé en utilisant une nouvelle classe et de nouvelles méthodes.

\subsubsection{\texorpdfstring{\textbf{Définition}}{Définition}}\label{duxe9finition}

   \begin{code}\begin{Verbatim}[commandchars=\\\{\}]
{\color{incolor}In [{\color{incolor}0}]:} \PY{k}{class} \PY{n+nc}{IntegerModRing}\PY{p}{:} \PY{c+c1}{\PYZsh{}Z/nZ}
          
          \PY{k}{def} \PY{n+nf}{\PYZus{}\PYZus{}init\PYZus{}\PYZus{}}\PY{p}{(}\PY{n+nb+bp}{self}\PY{p}{,} \PY{n}{n}\PY{p}{)}\PY{p}{:} \PY{c+c1}{\PYZsh{} Constructeur : IntegerModRing(n)}
            \PY{n+nb+bp}{self}\PY{o}{.}\PY{n}{n} \PY{o}{=} \PY{n}{n}
            \PY{n+nb+bp}{self}\PY{o}{.}\PY{n+nf+fm}{\PYZus{}\PYZus{}call\PYZus{}\PYZus{}}\PY{o}{.}\PY{n}{n} \PY{o}{=} \PY{n+nb+bp}{self}\PY{o}{.}\PY{n}{n}
        
          \PY{k}{class} \PY{n+nc}{\PYZus{}\PYZus{}call\PYZus{}\PYZus{}}\PY{p}{:} \PY{c+c1}{\PYZsh{} Rendre l\PYZsq{}objet \PYZdq{}Callable\PYZdq{} : A(x)}
              \PY{k}{def} \PY{n+nf}{\PYZus{}\PYZus{}init\PYZus{}\PYZus{}}\PY{p}{(}\PY{n+nb+bp}{self}\PY{p}{,} \PY{n}{x}\PY{p}{)}\PY{p}{:}
                \PY{n+nb+bp}{self}\PY{o}{.}\PY{n}{value} \PY{o}{=} \PY{n}{x} \PY{o}{\PYZpc{}} \PY{n+nb+bp}{self}\PY{o}{.}\PY{n}{n}
              \PY{k}{def} \PY{n+nf}{\PYZus{}\PYZus{}add\PYZus{}\PYZus{}}\PY{p}{(}\PY{n+nb+bp}{self}\PY{p}{,} \PY{n}{a}\PY{p}{)}\PY{p}{:} \PY{c+c1}{\PYZsh{} Redéfinition de l'addition}
                \PY{k}{return} \PY{p}{(}\PY{n+nb+bp}{self}\PY{o}{.}\PY{n}{value}\PY{o}{+}\PY{n}{a}\PY{p}{)} \PY{o}{\PYZpc{}} \PY{n+nb+bp}{self}\PY{o}{.}\PY{n}{n}
              \PY{k}{def} \PY{n+nf}{\PYZus{}\PYZus{}mul\PYZus{}\PYZus{}}\PY{p}{(}\PY{n+nb+bp}{self}\PY{p}{,} \PY{n}{multi}\PY{p}{)}\PY{p}{:} \PY{c+c1}{\PYZsh{} Redéfinition de la multiplication}
                \PY{k}{return} \PY{p}{(}\PY{n+nb+bp}{self}\PY{o}{.}\PY{n}{value} \PY{o}{*} \PY{n}{multi}\PY{p}{)} \PY{o}{\PYZpc{}} \PY{n+nb+bp}{self}\PY{o}{.}\PY{n}{n}
              \PY{k}{def} \PY{n+nf}{\PYZus{}\PYZus{}pow\PYZus{}\PYZus{}}\PY{p}{(}\PY{n+nb+bp}{self}\PY{p}{,} \PY{n}{power}\PY{p}{)}\PY{p}{:} \PY{c+c1}{\PYZsh{} Redéfinition de la puissance}
                \PY{k}{return} \PY{n}{expoRapide}\PY{p}{(}\PY{n+nb+bp}{self}\PY{o}{.}\PY{n}{value}\PY{p}{,} \PY{n}{power}\PY{p}{,} \PY{n+nb+bp}{self}\PY{o}{.}\PY{n}{n}\PY{p}{)}
\end{Verbatim}
\end{code}


\subsubsection{\texorpdfstring{\textbf{Test}}{Test}}\label{test}

   \begin{code}\begin{Verbatim}[commandchars=\\\{\}]
{\color{incolor}In [{\color{incolor}193}]:} \PY{n}{x} \PY{o}{=} \PY{l+m+mi}{11}
          \PY{n}{k} \PY{o}{=} \PY{l+m+mi}{2}
          
          \PY{n}{n} \PY{o}{=} \PY{l+m+mi}{5}
          
          \PY{n+nb}{print}\PY{p}{(}\PY{l+s+s2}{\PYZdq{}}\PY{l+s+s2}{A = IntegerModRing(}\PY{l+s+s2}{\PYZdq{}}\PY{p}{,}\PY{n}{n}\PY{p}{,}\PY{l+s+s2}{\PYZdq{}}\PY{l+s+s2}{)}\PY{l+s+s2}{\PYZdq{}}\PY{p}{)}
          \PY{n}{A} \PY{o}{=} \PY{n}{IntegerModRing}\PY{p}{(}\PY{n}{n}\PY{p}{)}
          \PY{n+nb}{print}\PY{p}{(}\PY{l+s+s2}{\PYZdq{}}\PY{l+s+s2}{\PYZdq{}}\PY{p}{)}
          
          \PY{n}{r} \PY{o}{=} \PY{n}{A}\PY{p}{(}\PY{n}{x}\PY{p}{)}\PY{o}{+}\PY{n}{k}
          \PY{n+nb}{print}\PY{p}{(}\PY{l+s+s2}{\PYZdq{}}\PY{l+s+s2}{Addition A(x)+k :}\PY{l+s+se}{\PYZbs{}n}\PY{l+s+s2}{\PYZdq{}}\PY{p}{,}\PY{n}{x}\PY{p}{,}\PY{l+s+s2}{\PYZdq{}}\PY{l+s+s2}{+}\PY{l+s+s2}{\PYZdq{}}\PY{p}{,}\PY{p}{k}\PY{p}{,}\PY{l+s+s2}{\PYZdq{}}\PY{l+s+s2}{\PYZpc{}}\PY{l+s+s2}{\PYZdq{}}\PY{p}{,}\PY{n}{n}\PY{p}{,}\PY{l+s+s2}{\PYZdq{}}\PY{l+s+s2}{=}\PY{l+s+s2}{\PYZdq{}}\PY{p}{,}\PY{n}{r}\PY{p}{)}
          \PY{n+nb}{print}\PY{p}{(}\PY{l+s+s2}{\PYZdq{}}\PY{l+s+s2}{\PYZdq{}}\PY{p}{)}
          
          \PY{n}{r} \PY{o}{=} \PY{n}{A}\PY{p}{(}\PY{n}{x}\PY{p}{)}\PY{o}{*}\PY{n}{k}
          \PY{n+nb}{print}\PY{p}{(}\PY{l+s+s2}{\PYZdq{}}\PY{l+s+s2}{Multiplication A(x)**k :}\PY{l+s+se}{\PYZbs{}n}\PY{l+s+s2}{\PYZdq{}}\PY{p}{,}\PY{n}{x}\PY{p}{,}\PY{l+s+s2}{\PYZdq{}}\PY{l+s+s2}{*}\PY{l+s+s2}{\PYZdq{}}\PY{p}{,}\PY{n}{k}\PY{p}{,}\PY{l+s+s2}{\PYZdq{}}\PY{l+s+s2}{\PYZpc{}}\PY{l+s+s2}{\PYZdq{}}\PY{p}{,}\PY{n}{n}\PY{p}{,}\PY{l+s+s2}{\PYZdq{}}\PY{l+s+s2}{=}\PY{l+s+s2}{\PYZdq{}}\PY{p}{,}\PY{n}{r}\PY{p}{)}
          \PY{n+nb}{print}\PY{p}{(}\PY{l+s+s2}{\PYZdq{}}\PY{l+s+s2}{\PYZdq{}}\PY{p}{)}
          
          \PY{n}{r} \PY{o}{=} \PY{n}{A}\PY{p}{(}\PY{n}{x}\PY{p}{)}\PY{o}{*}\PY{o}{*}\PY{n}{k}
          \PY{n+nb}{print}\PY{p}{(}\PY{l+s+s2}{\PYZdq{}}\PY{l+s+s2}{Puissance A(x)*k :}\PY{l+s+se}{\PYZbs{}n}\PY{l+s+s2}{\PYZdq{}}\PY{p}{,}\PY{n}{x}\PY{p}{,}\PY{l+s+s2}{\PYZdq{}}\PY{l+s+s2}{\PYZca{}}\PY{l+s+s2}{\PYZdq{}}\PY{p}{,}\PY{n}{k}\PY{p}{,}\PY{l+s+s2}{\PYZdq{}}\PY{l+s+s2}{\PYZpc{}}\PY{l+s+s2}{\PYZdq{}}\PY{p}{,}\PY{n}{n}\PY{p}{,}\PY{l+s+s2}{\PYZdq{}}\PY{l+s+s2}{=}\PY{l+s+s2}{\PYZdq{}}\PY{p}{,}\PY{n}{r}\PY{p}{)}
          \PY{n+nb}{print}\PY{p}{(}\PY{l+s+s2}{\PYZdq{}}\PY{l+s+s2}{\PYZdq{}}\PY{p}{)}
          
          \PY{n}{r} \PY{o}{=} \PY{n}{A}\PY{p}{(}\PY{n}{x}\PY{p}{)}\PY{o}{*}\PY{o}{*}\PY{p}{(}\PY{o}{\PYZhy{}}\PY{l+m+mi}{1}\PY{p}{)}
          \PY{n+nb}{print}\PY{p}{(}\PY{l+s+s2}{\PYZdq{}}\PY{l+s+s2}{Inverse A(x)**(\PYZhy{}1) :}\PY{l+s+se}{\PYZbs{}n}\PY{l+s+s2}{\PYZdq{}}\PY{p}{,}\PY{n}{x}\PY{p}{,}\PY{l+s+s2}{\PYZdq{}}\PY{l+s+s2}{\PYZca{} \PYZhy{}1 }\PY{l+s+s2}{\PYZpc{}}\PY{l+s+s2}{\PYZdq{}}\PY{p}{,}\PY{n}{n}\PY{p}{,}\PY{l+s+s2}{\PYZdq{}}\PY{l+s+s2}{=}\PY{l+s+s2}{\PYZdq{}}\PY{p}{,}\PY{n}{r}\PY{p}{)}
\end{Verbatim}
\end{code}


   \begin{code}\begin{Verbatim}[commandchars=\\\{\}]
A = IntegerModRing( 5 )

Addition A(x)+k :
 11 + 2 \% 5 = 3

Multiplication A(x)**k :
 11 * 2 \% 5 = 2

Puissance A(x)*k :
 11 \^{} 2 \% 5 = 1

Inverse A(x)**(-1) :
 11 \^{} -1 \% 5 = 1

    \end{Verbatim}
\end{code}

\chapter{\texorpdfstring{\textbf{Partie 2 - Codage et
Décodage}}{Partie 2 - Codage et Décodage}}\label{partie-2---codage-et-duxe9codage}

\section{\texorpdfstring{\textbf{Question
1}}{Question 1}}\label{question-1}

\subsection{\texorpdfstring{\textbf{Alphabet et
Dictionnaires}}{Alphabet et Dictionnaires}}\label{alphabet-et-dictionnaires}

\subsubsection{\texorpdfstring{\textbf{Définition}}{Définition}}\label{duxe9finition}

   \begin{code}\begin{Verbatim}[commandchars=\\\{\}]
{\color{incolor}In [{\color{incolor}0}]:} \PY{n}{abc} \PY{o}{=} \PY{l+s+s2}{\PYZdq{}}\PY{l+s+s2}{\PYZdl{}}\PY{l+s+se}{\PYZbs{}n}\PY{l+s+s2}{ abcdefghijklmnopqrstuvwxyzABCDEFGHIJKLMNOPQRSTUVWXYZ1234567890,.;:!?}\PY{l+s+s2}{\PYZsq{}}\PY{l+s+s2}{()\PYZam{}}\PY{l+s+s2}{\PYZpc{}}\PY{l+s+s2}{\PYZhy{}+*/=@}\PY{l+s+s2}{\PYZdq{}}
        
        \PY{n}{cDico} \PY{o}{=} \PY{p}{\PYZob{}}\PY{n}{abc}\PY{p}{[}\PY{n}{key}\PY{p}{]}\PY{p}{:}\PY{n}{key} \PY{k}{for} \PY{n}{key} \PY{o+ow}{in} \PY{n+nb}{range}\PY{p}{(}\PY{n+nb}{len}\PY{p}{(}\PY{n}{abc}\PY{p}{)}\PY{p}{)}\PY{p}{\PYZcb{}}
        \PY{n}{dDico} \PY{o}{=} \PY{p}{\PYZob{}}\PY{n}{key}\PY{p}{:}\PY{n}{abc}\PY{p}{[}\PY{n}{key}\PY{p}{]} \PY{k}{for} \PY{n}{key} \PY{o+ow}{in} \PY{n+nb}{range}\PY{p}{(}\PY{n+nb}{len}\PY{p}{(}\PY{n}{abc}\PY{p}{)}\PY{p}{)}\PY{p}{\PYZcb{}}
\end{Verbatim}
\end{code}


\subsubsection{\texorpdfstring{\textbf{Test}}{Test}}\label{test}

   \begin{code}\begin{Verbatim}[commandchars=\\\{\}]
{\color{incolor}In [{\color{incolor}195}]:} \PY{n+nb}{print}\PY{p}{(}\PY{l+s+s2}{\PYZdq{}}\PY{l+s+s2}{Début }\PY{l+s+se}{\PYZbs{}\PYZdq{}}\PY{l+s+s2}{\PYZdq{}}\PY{p}{,}\PY{n}{abc}\PY{p}{[}\PY{l+m+mi}{0}\PY{p}{]}\PY{p}{,} \PY{l+s+s2}{\PYZdq{}}\PY{l+s+s2}{\PYZhy{}}\PY{l+s+s2}{\PYZdq{}}\PY{p}{,} \PY{n}{abc}\PY{p}{[}\PY{l+m+mi}{1}\PY{p}{]}\PY{p}{,}\PY{l+s+s2}{\PYZdq{}}\PY{l+s+se}{\PYZbs{}\PYZdq{}}\PY{l+s+s2}{ Fin}\PY{l+s+s2}{\PYZdq{}}\PY{p}{)}
          \PY{n+nb}{print}\PY{p}{(}\PY{l+s+s2}{\PYZdq{}}\PY{l+s+s2}{Tailles :}\PY{l+s+s2}{\PYZdq{}}\PY{p}{,} \PY{n+nb}{len}\PY{p}{(}\PY{n}{abc}\PY{p}{)}\PY{p}{,} \PY{n+nb}{len}\PY{p}{(}\PY{n}{cDico}\PY{p}{)}\PY{p}{,} \PY{n+nb}{len}\PY{p}{(}\PY{n}{dDico}\PY{p}{)}\PY{p}{)}
          \PY{n+nb}{print}\PY{p}{(}\PY{l+s+s2}{\PYZdq{}}\PY{l+s+s2}{\PYZdq{}}\PY{p}{)}
          
          \PY{n+nb}{print}\PY{p}{(}\PY{l+s+s2}{\PYZdq{}}\PY{l+s+s2}{Alphabet :}\PY{l+s+s2}{\PYZdq{}}\PY{p}{,}\PY{n}{abc}\PY{p}{)}
          \PY{n+nb}{print}\PY{p}{(}\PY{l+s+s2}{\PYZdq{}}\PY{l+s+s2}{\PYZdq{}}\PY{p}{)}
          \PY{n+nb}{print}\PY{p}{(}\PY{l+s+s2}{\PYZdq{}}\PY{l+s+s2}{Dico de codage :}\PY{l+s+s2}{\PYZdq{}}\PY{p}{,}\PY{n}{cDico}\PY{p}{)}
          \PY{n+nb}{print}\PY{p}{(}\PY{l+s+s2}{\PYZdq{}}\PY{l+s+s2}{Dico de décodage :}\PY{l+s+s2}{\PYZdq{}}\PY{p}{,}\PY{n}{dDico}\PY{p}{)}
\end{Verbatim}
\end{code}


   \begin{code}\begin{Verbatim}[commandchars=\\\{\}]
Début " \$ - 
 " Fin
Tailles : 82 82 82

Alphabet : \$
 abcdefghijklmnopqrstuvwxyzABCDEFGHIJKLMNOPQRSTUVWXYZ1234567890,.;:!?'()\&\%-+*/=@

Dico de codage : \{'\$': 0, '\textbackslash{}n': 1, ' ': 2, 'a': 3, 'b': 4, 'c': 5, 'd': 6, 'e': 7, 'f': 8, 'g': 9,
                  ......
                  ')': 73, '\&': 74, '\%': 75, '-': 76, '+': 77, '*': 78, '/': 79, '=': 80, '@': 81\}
                  
Dico de décodage : \{0: '\$', 1: '\textbackslash{}n', 2: ' ', 3: 'a', 4: 'b', 5: 'c', 6: 'd', 7: 'e', 8: 'f', 9: 'g',
                    ......
                    73: ')', 74: '\&', 75: '\%', 76: '-', 77: '+', 78: '*', 79: '/', 80: '=', 81: '@'\}

    \end{Verbatim}
\end{code}

\section{\texorpdfstring{\textbf{Question
2}}{Question 2}}\label{question-2}

\subsection{\texorpdfstring{\textbf{Encodage d'un
bloc}}{Encodage d'un bloc}}\label{encodage-dun-bloc}

\subsubsection{\texorpdfstring{\textbf{Définition}}{Définition}}\label{duxe9finition}

Remarque : Dans le cas d'un caractère inconnu, il est remplacé par le
dernier caractère de l'alphabet, rajouté pour l'occasion : "@".

   \begin{code}\begin{Verbatim}[commandchars=\\\{\}]
{\color{incolor}In [{\color{incolor}0}]:} \PY{k}{def} \PY{n+nf}{encodageBloc}\PY{p}{(}\PY{n}{bloc}\PY{p}{)}\PY{p}{:}
          
          \PY{n}{result} \PY{o}{=} \PY{p}{[}\PY{p}{]}
          \PY{k}{for} \PY{n}{c} \PY{o+ow}{in} \PY{n}{bloc}\PY{p}{:}
            \PY{k}{try}\PY{p}{:}
              \PY{n}{result}\PY{o}{.}\PY{n}{append}\PY{p}{(}\PY{n}{cDico}\PY{p}{[}\PY{n}{c}\PY{p}{]}\PY{p}{)}
            \PY{k}{except} \PY{n+ne}{KeyError}\PY{p}{:}
              \PY{n}{result}\PY{o}{.}\PY{n}{append}\PY{p}{(}\PY{n+nb}{len}\PY{p}{(}\PY{n}{abc}\PY{p}{)}\PY{o}{\PYZhy{}}\PY{l+m+mi}{1}\PY{p}{)}
        
          \PY{n}{number} \PY{o}{=} \PY{l+m+mi}{0}
          \PY{k}{for} \PY{n}{i} \PY{o+ow}{in} \PY{n+nb}{range}\PY{p}{(}\PY{n+nb}{len}\PY{p}{(}\PY{n}{result}\PY{p}{)}\PY{p}{)}\PY{p}{:}
            \PY{n}{number} \PY{o}{+}\PY{o}{=} \PY{n}{result}\PY{p}{[}\PY{n}{i}\PY{p}{]}\PY{o}{*}\PY{n+nb}{len}\PY{p}{(}\PY{n}{abc}\PY{p}{)}\PY{o}{*}\PY{o}{*}\PY{n}{i}
        
          \PY{k}{return} \PY{n}{number}
\end{Verbatim}
\end{code}


\subsubsection{\texorpdfstring{\textbf{Test}}{Test}}\label{test}

   \begin{code}\begin{Verbatim}[commandchars=\\\{\}]
{\color{incolor}In [{\color{incolor}197}]:} \PY{n}{msg} \PY{o}{=} \PY{l+s+s2}{\PYZdq{}}\PY{l+s+s2}{Hello World}\PY{l+s+s2}{\PYZdq{}}
          \PY{n+nb}{print}\PY{p}{(}\PY{l+s+s2}{\PYZdq{}}\PY{l+s+s2}{Le message est : }\PY{l+s+se}{\PYZbs{}\PYZdq{}}\PY{l+s+s2}{\PYZdq{}}\PY{o}{+}\PY{n}{msg}\PY{o}{+}\PY{l+s+s2}{\PYZdq{}}\PY{l+s+se}{\PYZbs{}\PYZdq{}}\PY{l+s+s2}{\PYZdq{}}\PY{p}{)}
          
          \PY{n+nb}{print}\PY{p}{(}\PY{l+s+s2}{\PYZdq{}}\PY{l+s+s2}{\PYZdq{}}\PY{p}{)}
          \PY{n}{cMsg} \PY{o}{=} \PY{n}{encodageBloc}\PY{p}{(}\PY{n}{msg}\PY{p}{)}
          \PY{n+nb}{print}\PY{p}{(}\PY{l+s+s2}{\PYZdq{}}\PY{l+s+s2}{L}\PY{l+s+s2}{\PYZsq{}}\PY{l+s+s2}{encodage du message donne :}\PY{l+s+s2}{\PYZdq{}}\PY{p}{,} \PY{n}{cMsg}\PY{p}{)}
\end{Verbatim}
\end{code}


   \begin{code}\begin{Verbatim}[commandchars=\\\{\}]
Le message est : "Hello World"

L'encodage du message donne : 84856814622316789146

    \end{Verbatim}
\end{code}

\section{\texorpdfstring{\textbf{Question
3}}{Question 3}}\label{question-3}

Pour une taille de bloc \(P\), et un alphabet de taille \(N\), le code
numérique maximal d'un bloc est de \(N^{P-1}\)

\subsection{\texorpdfstring{\textbf{Décodage d'un
bloc}}{Décodage d'un bloc}}\label{duxe9codage-dun-bloc}

\subsubsection{\texorpdfstring{\textbf{Définition}}{Définition}}\label{duxe9finition}

Remarque : Une taille de bloc est nécessaire en paramètre pour connaitre
la puissance maximale de l'encodage du dit-bloc.

   \begin{code}\begin{Verbatim}[commandchars=\\\{\}]
{\color{incolor}In [{\color{incolor}0}]:} \PY{k}{def} \PY{n+nf}{decodageBloc}\PY{p}{(}\PY{n}{bloc}\PY{p}{,} \PY{n}{maxChar}\PY{o}{=}\PY{l+m+mi}{20}\PY{p}{)}\PY{p}{:}
          \PY{k}{assert} \PY{n}{maxChar}\PY{o}{\PYZlt{}}\PY{n+nb}{len}\PY{p}{(}\PY{n}{abc}\PY{p}{)}\PY{p}{,} \PY{l+s+s2}{\PYZdq{}}\PY{l+s+s2}{Taille de bloc trop grande}\PY{l+s+s2}{\PYZdq{}}
          \PY{k}{assert} \PY{n}{bloc}\PY{o}{\PYZlt{}}\PY{p}{(}\PY{n+nb}{len}\PY{p}{(}\PY{n}{abc}\PY{p}{)}\PY{o}{*}\PY{o}{*}\PY{p}{(}\PY{n}{maxChar}\PY{p}{)}\PY{p}{)}\PY{p}{,} \PY{l+s+s2}{\PYZdq{}}\PY{l+s+s2}{Bloc non valide, valeur trop grande}\PY{l+s+s2}{\PYZdq{}}
        
          \PY{n}{result} \PY{o}{=} \PY{p}{[}\PY{p}{]}
        
          \PY{n}{reste} \PY{o}{=} \PY{n}{bloc}
          \PY{k}{for} \PY{n}{i} \PY{o+ow}{in} \PY{n+nb}{range}\PY{p}{(}\PY{n}{maxChar}\PY{o}{\PYZhy{}}\PY{l+m+mi}{1}\PY{p}{,} \PY{o}{\PYZhy{}}\PY{l+m+mi}{1}\PY{p}{,} \PY{o}{\PYZhy{}}\PY{l+m+mi}{1}\PY{p}{)}\PY{p}{:}
            \PY{k}{if}\PY{p}{(}\PY{n}{reste} \PY{o}{\PYZgt{}}\PY{o}{=} \PY{n+nb}{len}\PY{p}{(}\PY{n}{abc}\PY{p}{)}\PY{o}{*}\PY{o}{*}\PY{n}{i}\PY{p}{)}\PY{p}{:}
              \PY{n}{result}\PY{o}{.}\PY{n}{append}\PY{p}{(}\PY{n}{reste}\PY{o}{/}\PY{o}{/}\PY{p}{(}\PY{n+nb}{len}\PY{p}{(}\PY{n}{abc}\PY{p}{)}\PY{o}{*}\PY{o}{*}\PY{n}{i}\PY{p}{)}\PY{p}{)}
              \PY{n}{reste} \PY{o}{=} \PY{n}{reste}\PY{o}{\PYZpc{}}\PY{p}{(}\PY{n+nb}{len}\PY{p}{(}\PY{n}{abc}\PY{p}{)}\PY{o}{*}\PY{o}{*}\PY{n}{i}\PY{p}{)}
        
          \PY{n}{msg} \PY{o}{=} \PY{l+s+s2}{\PYZdq{}}\PY{l+s+s2}{\PYZdq{}}
          \PY{k}{for} \PY{n}{n} \PY{o+ow}{in} \PY{n}{result}\PY{p}{:}
            \PY{k}{try}\PY{p}{:}
              \PY{n}{msg} \PY{o}{=} \PY{n}{dDico}\PY{p}{[}\PY{n}{n}\PY{p}{]} \PY{o}{+} \PY{n}{msg}
            \PY{k}{except} \PY{n+ne}{KeyError}\PY{p}{:}
              \PY{n}{msg} \PY{o}{=} \PY{l+s+s2}{\PYZdq{}}\PY{l+s+s2}{@}\PY{l+s+s2}{\PYZdq{}} \PY{o}{+} \PY{n}{msg}
        
          \PY{k}{return} \PY{n}{msg}
\end{Verbatim}
\end{code}


\subsubsection{\texorpdfstring{\textbf{Test}}{Test}}\label{test}

Ajout d'un caractère avant le décodage : ajout d'un \(len(dDico)\) (Afin
d'être toujours en dehors de l'alphabet).

   \begin{code}\begin{Verbatim}[commandchars=\\\{\}]
{\color{incolor}In [{\color{incolor}199}]:} \PY{n}{dMsg} \PY{o}{=} \PY{n}{decodageBloc}\PY{p}{(}\PY{n}{cMsg}\PY{p}{)}
          \PY{n+nb}{print}\PY{p}{(}\PY{l+s+s2}{\PYZdq{}}\PY{l+s+s2}{Le decodage du message donne : }\PY{l+s+se}{\PYZbs{}\PYZdq{}}\PY{l+s+s2}{\PYZdq{}}\PY{o}{+}\PY{n}{dMsg}\PY{o}{+}\PY{l+s+s2}{\PYZdq{}}\PY{l+s+se}{\PYZbs{}\PYZdq{}}\PY{l+s+s2}{\PYZdq{}}\PY{p}{)}
          \PY{n+nb}{print}\PY{p}{(}\PY{l+s+s2}{\PYZdq{}}\PY{l+s+s2}{                Vérification : }\PY{l+s+se}{\PYZbs{}\PYZdq{}}\PY{l+s+s2}{\PYZdq{}}\PY{o}{+}\PY{n}{msg}\PY{o}{+}\PY{l+s+s2}{\PYZdq{}}\PY{l+s+se}{\PYZbs{}\PYZdq{}}\PY{l+s+s2}{\PYZdq{}}\PY{p}{)}
\end{Verbatim}
\end{code}


   \begin{code}\begin{Verbatim}[commandchars=\\\{\}]
Le decodage du message donne : "Hello World"
                Vérification : "Hello World"

    \end{Verbatim}
\end{code}

\section{\texorpdfstring{\textbf{Question
4}}{Question 4}}\label{question-4}

\subsection{\texorpdfstring{\textbf{Encodage d'un message
entier}}{Encodage d'un message entier}}\label{encodage-dun-message-entier}

\subsubsection{\texorpdfstring{\textbf{Définition}}{Définition}}\label{duxe9finition}

   \begin{code}\begin{Verbatim}[commandchars=\\\{\}]
{\color{incolor}In [{\color{incolor}0}]:} \PY{k}{def} \PY{n+nf}{encodage}\PY{p}{(}\PY{n}{msg}\PY{p}{,} \PY{n}{tailleBloc}\PY{o}{=}\PY{l+m+mi}{20}\PY{p}{)}\PY{p}{:}
          
          \PY{n}{encodedMsg} \PY{o}{=} \PY{p}{[}\PY{p}{]}
        
          \PY{k}{for} \PY{n}{i} \PY{o+ow}{in} \PY{n+nb}{range}\PY{p}{(}\PY{l+m+mi}{0}\PY{p}{,} \PY{n+nb}{len}\PY{p}{(}\PY{n}{msg}\PY{p}{)}\PY{p}{,} \PY{n}{tailleBloc}\PY{p}{)}\PY{p}{:}
            \PY{n}{encodedMsg}\PY{o}{.}\PY{n}{append}\PY{p}{(}\PY{n}{encodageBloc}\PY{p}{(}\PY{n}{msg}\PY{p}{[}\PY{n}{i}\PY{p}{:}\PY{n}{i}\PY{o}{+}\PY{n}{tailleBloc}\PY{p}{]}\PY{p}{)}\PY{p}{)}
        
          \PY{k}{return} \PY{n}{encodedMsg}
\end{Verbatim}
\end{code}


\subsubsection{\texorpdfstring{\textbf{Test}}{Test}}\label{test}

   \begin{code}\begin{Verbatim}[commandchars=\\\{\}]
{\color{incolor}In [{\color{incolor}201}]:} \PY{n}{msg} \PY{o}{=} \PY{l+s+s2}{\PYZdq{}}\PY{l+s+s2}{à Lorem ipsum dolor sit amet, consectetur adipiscing elit. Quisque egestas eget orci}
                 \PY{l+s+s2}{sit amet ullamcorper. Curabitur venenatis non nulla eu ullamcorper. Sed dignissim,}
                 \PY{l+s+s2}{......}
                 \PY{l+s+s2}{tristique at. Morbi augue purus, elementum non posuere ac, pulvinar sit amet odio.}
                 \PY{l+s+s2}{Nulla vitae. }\PY{l+s+s2}{\PYZdq{}}
          \PY{n+nb}{print}\PY{p}{(}\PY{l+s+s2}{\PYZdq{}}\PY{l+s+s2}{Le message est : }\PY{l+s+se}{\PYZbs{}\PYZdq{}}\PY{l+s+s2}{\PYZdq{}}\PY{o}{+}\PY{n}{msg}\PY{o}{+}\PY{l+s+s2}{\PYZdq{}}\PY{l+s+se}{\PYZbs{}\PYZdq{}}\PY{l+s+s2}{\PYZdq{}}\PY{p}{)}
          
          \PY{n+nb}{print}\PY{p}{(}\PY{l+s+s2}{\PYZdq{}}\PY{l+s+s2}{\PYZdq{}}\PY{p}{)}
          \PY{n}{cMsg} \PY{o}{=} \PY{n}{encodage}\PY{p}{(}\PY{n}{msg}\PY{p}{)}
          \PY{n+nb}{print}\PY{p}{(}\PY{l+s+s2}{\PYZdq{}}\PY{l+s+s2}{L}\PY{l+s+s2}{\PYZsq{}}\PY{l+s+s2}{encodage du message complet donne :}\PY{l+s+s2}{\PYZdq{}}\PY{p}{,} \PY{n}{cMsg}\PY{p}{)}
\end{Verbatim}
\end{code}


   \begin{code}\begin{Verbatim}[commandchars=\\\{\}]
\textbf{Le message est :} "à Lorem ipsum dolor sit amet, consectetur adipiscing elit. Quisque egestas eget 
orci sit amet ullamcorper. Curabitur venenatis non nulla eu ullamcorper. Sed dignissim, libero sed
......
tristique at. Morbi augue purus, elementum non posuere ac, pulvinar sit amet odio. Nulla vitae. "

L'encodage du message complet donne : [5175605931744531058174221399946258109,
53610257152947054120140631808764154323, 103754296707586791015869521983086626148,
......
48444910900856704617692622269544623071, 4696941013398074923860390960090156871, 853019927550]

    \end{Verbatim}
\end{code}

\subsection{\texorpdfstring{\textbf{Décodage d'un message codé entier
(soit une liste de blocs
encodés)}}{Décodage d'un message codé entier (soit une liste de blocs encodés)}}\label{duxe9codage-dun-message-coduxe9-entier-soit-une-liste-de-blocs-encoduxe9s}

\subsubsection{\texorpdfstring{\textbf{Définition}}{Définition}}\label{duxe9finition}

   \begin{code}\begin{Verbatim}[commandchars=\\\{\}]
{\color{incolor}In [{\color{incolor}0}]:} \PY{k}{def} \PY{n+nf}{decodage}\PY{p}{(}\PY{n}{blocs}\PY{p}{,} \PY{n}{tailleBloc}\PY{o}{=}\PY{l+m+mi}{20}\PY{p}{)}\PY{p}{:}
          
          \PY{n}{decodedMsg} \PY{o}{=} \PY{l+s+s2}{\PYZdq{}}\PY{l+s+s2}{\PYZdq{}}
        
          \PY{k}{for} \PY{n}{bloc} \PY{o+ow}{in} \PY{n}{blocs}\PY{p}{:}
            \PY{n}{decodedMsg} \PY{o}{+}\PY{o}{=} \PY{n}{decodageBloc}\PY{p}{(}\PY{n}{bloc}\PY{p}{,} \PY{n}{tailleBloc}\PY{p}{)}
        
          \PY{k}{return} \PY{n}{decodedMsg}
\end{Verbatim}
\end{code}


\subsubsection{\texorpdfstring{\textbf{Test}}{Test}}\label{test}

Remarque : Le caractère special "à" à été remplacé par un "@" comme
prévu

   \begin{code}\begin{Verbatim}[commandchars=\\\{\}]
{\color{incolor}In [{\color{incolor}203}]:} \PY{n}{dMsg} \PY{o}{=} \PY{n}{decodage}\PY{p}{(}\PY{n}{cMsg}\PY{p}{)}
          \PY{n+nb}{print}\PY{p}{(}\PY{l+s+s2}{\PYZdq{}}\PY{l+s+s2}{Le decodage du message complet donne : }\PY{l+s+se}{\PYZbs{}\PYZdq{}}\PY{l+s+s2}{\PYZdq{}}\PY{o}{+}\PY{n}{dMsg}\PY{o}{+}\PY{l+s+s2}{\PYZdq{}}\PY{l+s+se}{\PYZbs{}\PYZdq{}}\PY{l+s+s2}{\PYZdq{}}\PY{p}{)}
          \PY{n+nb}{print}\PY{p}{(}\PY{l+s+s2}{\PYZdq{}}\PY{l+s+s2}{                        Vérification : }\PY{l+s+se}{\PYZbs{}\PYZdq{}}\PY{l+s+s2}{\PYZdq{}}\PY{o}{+}\PY{n}{msg}\PY{o}{+}\PY{l+s+s2}{\PYZdq{}}\PY{l+s+se}{\PYZbs{}\PYZdq{}}\PY{l+s+s2}{\PYZdq{}}\PY{p}{)}
\end{Verbatim}
\end{code}


   \begin{code}\begin{Verbatim}[commandchars=\\\{\}]
Le decodage du message complet donne : "@ Lorem ipsum dolor sit amet, consectetur adipiscing elit. 
Quisque egestas eget orci sit amet ullamcorper. Curabitur venenatis non nulla eu ullamcorper
......
tristique at. Morbi augue purus, elementum non posuere ac, pulvinar sit amet odio. Nulla vitae. "

                        Vérification : "à Lorem ipsum dolor sit amet, consectetur adipiscing elit. 
Quisque egestas eget orci sit amet ullamcorper. Curabitur venenatis non nulla eu ullamcorper
......
tristique at. Morbi augue purus, elementum non posuere ac, pulvinar sit amet odio. Nulla vitae. "

    \end{Verbatim}
\end{code}

\chapter{\texorpdfstring{\textbf{Partie 3 -
RSA}}{Partie 3 - RSA}}\label{partie-3---rsa}

\section{\texorpdfstring{\textbf{Question
1}}{Question 1}}\label{question-1}

\subsection{\texorpdfstring{\textbf{Production des valeurs p, q et
a}}{Production des valeurs p, q et a}}\label{production-des-valeurs-p-q-et-a}

\subsubsection{\texorpdfstring{\textbf{Définition}}{Définition}}\label{duxe9finition}

   \begin{code}\begin{Verbatim}[commandchars=\\\{\}]
{\color{incolor}In [{\color{incolor}0}]:} \PY{k}{def} \PY{n+nf}{bobPrive}\PY{p}{(}\PY{n+nb}{min}\PY{o}{=}\PY{l+m+mi}{10}\PY{o}{*}\PY{o}{*}\PY{l+m+mi}{19}\PY{p}{,} \PY{n+nb}{max}\PY{o}{=}\PY{l+m+mi}{10}\PY{o}{*}\PY{o}{*}\PY{l+m+mi}{20}\PY{p}{)}\PY{p}{:}
        
          \PY{n}{p}\PY{p}{,} \PY{n}{q} \PY{o}{=} \PY{l+m+mi}{0}\PY{p}{,} \PY{l+m+mi}{0}
          \PY{k}{while} \PY{n}{q}\PY{o}{==}\PY{n}{p}\PY{p}{:}
            \PY{n}{p}\PY{p}{,} \PY{n}{q} \PY{o}{=} \PY{n}{sy}\PY{o}{.}\PY{n}{randprime}\PY{p}{(}\PY{n+nb}{min}\PY{p}{,} \PY{n+nb}{max}\PY{p}{)}\PY{p}{,} \PY{n}{sy}\PY{o}{.}\PY{n}{randprime}\PY{p}{(}\PY{n+nb}{min}\PY{p}{,} \PY{n+nb}{max}\PY{p}{)}
        
          \PY{k}{assert} \PY{p}{(}\PY{n+nb}{min}\PY{o}{\PYZgt{}}\PY{l+m+mi}{0} \PY{o+ow}{and} \PY{n+nb}{max} \PY{o}{\PYZgt{}} \PY{n+nb}{min}\PY{p}{)}\PY{p}{,} \PY{l+s+s2}{\PYZdq{}}\PY{l+s+s2}{Le paramêtre min doit être supérieur à 0 et max doit être}
                                         \PY{l+s+s2}{supérieur à min}\PY{l+s+s2}{\PYZdq{}}
        
          \PY{n}{a}\PY{p}{,} \PY{n}{r} \PY{o}{=} \PY{n}{p}\PY{p}{,} \PY{l+m+mi}{0}
          \PY{k}{while} \PY{n}{r}\PY{o}{!=}\PY{l+m+mi}{1}\PY{p}{:}
            \PY{n}{a} \PY{o}{=} \PY{n}{rand}\PY{o}{.}\PY{n}{randint}\PY{p}{(}\PY{n+nb}{min}\PY{p}{,} \PY{p}{(}\PY{n}{p}\PY{o}{\PYZhy{}}\PY{l+m+mi}{1}\PY{p}{)}\PY{o}{*}\PY{p}{(}\PY{n}{q}\PY{o}{\PYZhy{}}\PY{l+m+mi}{1}\PY{p}{)}\PY{o}{\PYZhy{}}\PY{l+m+mi}{1}\PY{p}{)}
            \PY{n}{r}\PY{p}{,} \PY{n}{\PYZus{}}\PY{p}{,} \PY{n}{\PYZus{}} \PY{o}{=} \PY{n}{euclideSimple}\PY{p}{(}\PY{n}{a}\PY{p}{,} \PY{p}{(}\PY{n}{p}\PY{o}{\PYZhy{}}\PY{l+m+mi}{1}\PY{p}{)}\PY{o}{*}\PY{p}{(}\PY{n}{q}\PY{o}{\PYZhy{}}\PY{l+m+mi}{1}\PY{p}{)}\PY{p}{)}
          
          \PY{k}{return} \PY{n}{p}\PY{p}{,} \PY{n}{q}\PY{p}{,} \PY{n}{a}
\end{Verbatim}
\end{code}


\subsubsection{\texorpdfstring{\textbf{Test}}{Test}}\label{test}

\paragraph{\texorpdfstring{\textbf{Partie
Privé}}{Partie Privé}}\label{partie-privuxe9}

Si \(power>14\), la fonction factor va devenir très chronophage.

   \begin{code}\begin{Verbatim}[commandchars=\\\{\}]
{\color{incolor}In [{\color{incolor}205}]:} \PY{n}{power} \PY{o}{=} \PY{l+m+mi}{19}
          
          \PY{n}{q}\PY{p}{,} \PY{n}{p}\PY{p}{,} \PY{n}{a} \PY{o}{=} \PY{n}{bobPrive}\PY{p}{(}\PY{l+m+mi}{10}\PY{o}{*}\PY{o}{*}\PY{n}{power}\PY{p}{,} \PY{l+m+mi}{10}\PY{o}{*}\PY{o}{*}\PY{p}{(}\PY{n}{power}\PY{o}{+}\PY{l+m+mi}{1}\PY{p}{)}\PY{p}{)}
          \PY{n+nb}{print}\PY{p}{(}\PY{l+s+s2}{\PYZdq{}}\PY{l+s+s2}{Bob produit 2 nombres premiers :}\PY{l+s+s2}{\PYZdq{}}\PY{p}{)}
          \PY{n+nb}{print}\PY{p}{(}\PY{l+s+s2}{\PYZdq{}}\PY{l+s+s2}{p =}\PY{l+s+s2}{\PYZdq{}}\PY{p}{,}\PY{n}{p}\PY{p}{,}\PY{l+s+s2}{\PYZdq{}}\PY{l+s+s2}{et q =}\PY{l+s+s2}{\PYZdq{}}\PY{p}{,}\PY{n}{q}\PY{p}{)}
          \PY{n+nb}{print}\PY{p}{(}\PY{l+s+s2}{\PYZdq{}}\PY{l+s+s2}{\PYZdq{}}\PY{p}{)}
          
          \PY{n+nb}{print}\PY{p}{(}\PY{l+s+s2}{\PYZdq{}}\PY{l+s+s2}{Bob produit un nombre aléatoire premier avec phi(n)=(p\PYZhy{}1)(q\PYZhy{}1) :}\PY{l+s+s2}{\PYZdq{}}\PY{p}{)}
          \PY{n+nb}{print}\PY{p}{(}\PY{l+s+s2}{\PYZdq{}}\PY{l+s+s2}{a =}\PY{l+s+s2}{\PYZdq{}}\PY{p}{,}\PY{n}{a}\PY{p}{)}
\end{Verbatim}
\end{code}


   \begin{code}\begin{Verbatim}[commandchars=\\\{\}]
Bob produit 2 nombres premiers :
p = 70839248108601181661 et q = 55524790103495060851

Bob produit un nombre aléatoire premier avec phi(n)=(p-1)(q-1) :
a = 820408104386500733550364758094963808171

    \end{Verbatim}
\end{code}

\paragraph{\texorpdfstring{\textbf{Partie
Public}}{Partie Public}}\label{partie-public}

   \begin{code}\begin{Verbatim}[commandchars=\\\{\}]
{\color{incolor}In [{\color{incolor}206}]:} \PY{n}{n} \PY{o}{=} \PY{n}{p}\PY{o}{*}\PY{n}{q}
          \PY{n+nb}{print}\PY{p}{(}\PY{l+s+s2}{\PYZdq{}}\PY{l+s+s2}{Bob publie n=pq :}\PY{l+s+s2}{\PYZdq{}}\PY{p}{)}
          \PY{n+nb}{print}\PY{p}{(}\PY{l+s+s2}{\PYZdq{}}\PY{l+s+s2}{n =}\PY{l+s+s2}{\PYZdq{}}\PY{p}{,}\PY{n}{n}\PY{p}{)}
          \PY{n+nb}{print}\PY{p}{(}\PY{l+s+s2}{\PYZdq{}}\PY{l+s+s2}{\PYZdq{}}\PY{p}{)}
          
          \PY{n}{b} \PY{o}{=} \PY{n}{inverseMod}\PY{p}{(}\PY{n}{a}\PY{p}{,} \PY{p}{(}\PY{n}{p}\PY{o}{\PYZhy{}}\PY{l+m+mi}{1}\PY{p}{)}\PY{o}{*}\PY{p}{(}\PY{n}{q}\PY{o}{\PYZhy{}}\PY{l+m+mi}{1}\PY{p}{)}\PY{p}{)}
          \PY{n+nb}{print}\PY{p}{(}\PY{l+s+s2}{\PYZdq{}}\PY{l+s+s2}{Bob publie l}\PY{l+s+s2}{\PYZsq{}}\PY{l+s+s2}{inverse de a modulo phi(n) :}\PY{l+s+s2}{\PYZdq{}}\PY{p}{)}
          \PY{n+nb}{print}\PY{p}{(}\PY{l+s+s2}{\PYZdq{}}\PY{l+s+s2}{b =}\PY{l+s+s2}{\PYZdq{}}\PY{p}{,}\PY{n}{b}\PY{p}{)}
\end{Verbatim}
\end{code}


   \begin{code}\begin{Verbatim}[commandchars=\\\{\}]
Bob publie n=pq :
n = 3933334382319490099117642142495700253511

Bob publie l'inverse de a modulo phi(n) :
b = 72445839558856047732133526800573386731

    \end{Verbatim}
\end{code}

\paragraph{\texorpdfstring{\textbf{Factor()}}{Factor()}}\label{factor}

La fonction utilisé ici est la fonction \(primefactors(n)\) de la
bibliothèque Sympy, le calcul est très chronophage pour des valeurs
supérieures à \(10^{15}\) (cf : Partie Privé/Test)

On remarque qu'il y a l'air d'y avoir que 2 facteurs premiers.

   \begin{code}\begin{Verbatim}[commandchars=\\\{\}]
{\color{incolor}In [{\color{incolor}207}]:} \PY{n}{power} \PY{o}{=} \PY{l+m+mi}{14}
          \PY{n}{q}\PY{p}{,} \PY{n}{p}\PY{p}{,} \PY{n}{a} \PY{o}{=} \PY{n}{bobPrive}\PY{p}{(}\PY{l+m+mi}{10}\PY{o}{*}\PY{o}{*}\PY{n}{power}\PY{p}{,} \PY{l+m+mi}{10}\PY{o}{*}\PY{o}{*}\PY{p}{(}\PY{n}{power}\PY{o}{+}\PY{l+m+mi}{1}\PY{p}{)}\PY{p}{)}
          \PY{n}{n} \PY{o}{=} \PY{n}{p}\PY{o}{*}\PY{n}{q}
          \PY{n+nb}{print}\PY{p}{(}\PY{n}{p,q}\PY{p}{)}
        
          \PY{k}{if} \PY{p}{(}\PY{n}{power} \PY{o}{\PYZlt{}}\PY{o}{=} \PY{l+m+mi}{15}\PY{p}{)}\PY{p}{:}
            \PY{n}{l} \PY{o}{=} \PY{n}{sy}\PY{o}{.}\PY{n}{primefactors}\PY{p}{(}\PY{n}{n}\PY{p}{)}
            \PY{n+nb}{print}\PY{p}{(}\PY{n}{l}\PY{p}{)}
          \PY{k}{else}\PY{p}{:}
            \PY{n+nb}{print}\PY{p}{(}\PY{l+s+s2}{\PYZdq{}}\PY{l+s+s2}{La valeur power est trop haute, le calcul durerait trop longtemps}\PY{l+s+s2}{\PYZdq{}}\PY{p}{)}
\end{Verbatim}
\end{code}


   \begin{code}\begin{Verbatim}[commandchars=\\\{\}]
934263701871971 808037973730219
[808037973730219, 934263701871971]

    \end{Verbatim}
\end{code}

\section{\texorpdfstring{\textbf{Question
2}}{Question 2}}\label{question-2}

\subsection{\texorpdfstring{\textbf{Obtention la taille du
bloc}}{Obtention la taille du bloc}}\label{obtention-la-taille-du-bloc}

\subsubsection{\texorpdfstring{\textbf{Définition}}{Définition}}\label{duxe9finition}

   \begin{code}\begin{Verbatim}[commandchars=\\\{\}]
{\color{incolor}In [{\color{incolor}0}]:} \PY{k}{def} \PY{n+nf}{tailleBlocMax}\PY{p}{(}\PY{n}{n}\PY{p}{)}\PY{p}{:}
        
          \PY{n}{tailleBloc} \PY{o}{=} \PY{l+m+mi}{0}
          \PY{k}{while} \PY{n+nb}{len}\PY{p}{(}\PY{n}{abc}\PY{p}{)}\PY{o}{*}\PY{o}{*}\PY{p}{(}\PY{n}{tailleBloc}\PY{o}{+}\PY{l+m+mi}{1}\PY{p}{)}\PY{o}{\PYZlt{}}\PY{n}{n}\PY{p}{:}
            \PY{n}{tailleBloc}\PY{o}{+}\PY{o}{=}\PY{l+m+mi}{1}
        
          \PY{k}{return} \PY{n}{tailleBloc}
\end{Verbatim}
\end{code}


\subsubsection{\texorpdfstring{\textbf{Test}}{Test}}\label{test}

   \begin{code}\begin{Verbatim}[commandchars=\\\{\}]
{\color{incolor}In [{\color{incolor}209}]:} \PY{n+nb}{print}\PY{p}{(}\PY{l+s+s2}{\PYZdq{}}\PY{l+s+s2}{Pour n =}\PY{l+s+s2}{\PYZdq{}}\PY{p}{,}\PY{n}{n}\PY{p}{,}\PY{l+s+s2}{\PYZdq{}}\PY{l+s+s2}{, la taille du bloc est de maximum :}\PY{l+s+s2}{\PYZdq{}}\PY{p}{,}\PY{n}{tailleBlocMax}\PY{p}{(}\PY{n}{n}\PY{p}{)}\PY{p}{)}
\end{Verbatim}
\end{code}


   \begin{code}\begin{Verbatim}[commandchars=\\\{\}]
Pour n = 3933334382319490099117642142495700253511 , la taille du bloc est de maximum : 20

    \end{Verbatim}
\end{code}

\subsection{\texorpdfstring{\textbf{Chiffrement RSA d'un simple
bloc}}{Chiffrement RSA d'un simple bloc}}\label{chiffrement-rsa-dun-simple-bloc}

\subsubsection{\texorpdfstring{\textbf{Définition}}{Définition}}\label{duxe9finition}

   \begin{code}\begin{Verbatim}[commandchars=\\\{\}]
{\color{incolor}In [{\color{incolor}0}]:} \PY{k}{def} \PY{n+nf}{chiffrementRSA}\PY{p}{(}\PY{n}{x}\PY{p}{,} \PY{n}{n}\PY{p}{,} \PY{n}{b}\PY{p}{)}\PY{p}{:}
          \PY{k}{assert} \PY{n}{x}\PY{o}{\PYZlt{}}\PY{n}{n}\PY{p}{,} \PY{l+s+s2}{\PYZdq{}}\PY{l+s+s2}{x doit être inférieur à n}\PY{l+s+s2}{\PYZdq{}}
        
          \PY{k}{return} \PY{n}{expoRapide}\PY{p}{(}\PY{n}{x}\PY{p}{,} \PY{n}{b}\PY{p}{,} \PY{n}{n}\PY{p}{)}
\end{Verbatim}
\end{code}


\subsubsection{\texorpdfstring{\textbf{Test}}{Test}}\label{test}

   \begin{code}\begin{Verbatim}[commandchars=\\\{\}]
{\color{incolor}In [{\color{incolor}211}]:} \PY{n}{msg} \PY{o}{=} \PY{l+s+s2}{\PYZdq{}}\PY{l+s+s2}{HelloWorld}\PY{l+s+s2}{\PYZdq{}}
          \PY{n}{cMsg} \PY{o}{=} \PY{n}{encodageBloc}\PY{p}{(}\PY{n}{msg}\PY{p}{)}
          \PY{n}{rsa} \PY{o}{=} \PY{n}{chiffrementRSA}\PY{p}{(}\PY{n}{cMsg}\PY{p}{,} \PY{n}{n}\PY{p}{,} \PY{n}{b}\PY{p}{)}
          \PY{n+nb}{print}\PY{p}{(}\PY{l+s+s2}{\PYZdq{}}\PY{l+s+s2}{Le message }\PY{l+s+se}{\PYZbs{}\PYZdq{}}\PY{l+s+s2}{\PYZdq{}}\PY{o}{+}\PY{n}{msg}\PY{o}{+}\PY{l+s+s2}{\PYZdq{}}\PY{l+s+se}{\PYZbs{}\PYZdq{}}\PY{l+s+s2}{ à été chiffré :}\PY{l+s+s2}{\PYZdq{}}\PY{p}{,}\PY{n}{rsa}\PY{p}{)}
\end{Verbatim}
\end{code}


   \begin{code}\begin{Verbatim}[commandchars=\\\{\}]
Le message "HelloWorld" à été chiffré : 3068046575213518710256046111929775664104

    \end{Verbatim}
\end{code}

\subsection{\texorpdfstring{\textbf{Chiffrement RSA d'un message
entier}}{Chiffrement RSA d'un message entier}}\label{chiffrement-rsa-dun-message-entier}

\subsubsection{\texorpdfstring{\textbf{Définition}}{Définition}}\label{duxe9finition}

   \begin{code}\begin{Verbatim}[commandchars=\\\{\}]
{\color{incolor}In [{\color{incolor}0}]:} \PY{k}{def} \PY{n+nf}{messageToRSA}\PY{p}{(}\PY{n}{msg}\PY{p}{,} \PY{n}{n}\PY{p}{,} \PY{n}{b}\PY{p}{)}\PY{p}{:}
        
          \PY{n}{taille} \PY{o}{=} \PY{n}{tailleBlocMax}\PY{p}{(}\PY{n}{n}\PY{p}{)}
          \PY{n}{l} \PY{o}{=} \PY{n}{encodage}\PY{p}{(}\PY{n}{msg}\PY{p}{,} \PY{n}{tailleBloc}\PY{o}{=}\PY{n}{taille}\PY{p}{)}
        
          \PY{n}{result} \PY{o}{=} \PY{p}{[}\PY{p}{]}
          \PY{k}{for} \PY{n}{elt} \PY{o+ow}{in} \PY{n}{l}\PY{p}{:}
            \PY{n}{result}\PY{o}{.}\PY{n}{append}\PY{p}{(}\PY{n}{chiffrementRSA}\PY{p}{(}\PY{n}{elt}\PY{p}{,} \PY{n}{n}\PY{p}{,} \PY{n}{b}\PY{p}{)}\PY{p}{)}
          
          \PY{k}{return} \PY{n}{result}
\end{Verbatim}
\end{code}


\subsubsection{\texorpdfstring{\textbf{Test}}{Test}}\label{test}

   \begin{code}\begin{Verbatim}[commandchars=\\\{\}]
{\color{incolor}In [{\color{incolor}213}]:} \PY{n}{msg} \PY{o}{=} \PY{l+s+s2}{\PYZdq{}}\PY{l+s+s2}{à Lorem ipsum dolor sit amet, consectetur adipiscing elit. Quisque egestas eget orci}
                 \PY{l+s+s2}{sit amet ullamcorper. Curabitur venenatis non nulla eu ullamcorper. Sed dignissim,}
                 \PY{l+s+s2}{......}
                 \PY{l+s+s2}{tristique at. Morbi augue purus, elementum non posuere ac, pulvinar sit amet odio.}
                 \PY{l+s+s2}{Nulla vitae. }\PY{l+s+s2}{\PYZdq{}}
                 
          \PY{n}{msgRSA} \PY{o}{=} \PY{n}{messageToRSA}\PY{p}{(}\PY{n}{msg}\PY{p}{,} \PY{n}{n}\PY{p}{,} \PY{n}{b}\PY{p}{)}
          
          \PY{n+nb}{print}\PY{p}{(}\PY{l+s+s2}{\PYZdq{}}\PY{l+s+s2}{Le message }\PY{l+s+se}{\PYZbs{}\PYZdq{}}\PY{l+s+s2}{\PYZdq{}}\PY{o}{+}\PY{n}{msg}\PY{o}{+}\PY{l+s+s2}{\PYZdq{}}\PY{l+s+se}{\PYZbs{}\PYZdq{}}\PY{l+s+s2}{\PYZdq{}}\PY{p}{)}
          \PY{n+nb}{print}\PY{p}{(}\PY{l+s+s2}{\PYZdq{}}\PY{l+s+s2}{à été chiffré :}\PY{l+s+s2}{\PYZdq{}}\PY{p}{,}\PY{n}{msgRSA}\PY{p}{)}
\end{Verbatim}
\end{code}


   \begin{code}\begin{Verbatim}[commandchars=\\\{\}]
\textbf{Le message} "à Lorem ipsum dolor sit amet, consectetur adipiscing elit. Quisque egestas eget orci sit 
amet ullamcorper. Curabitur venenatis non nulla eu ullamcorper. Sed dignissim, libero sed sodales
......
tristique at. Morbi augue purus, elementum non posuere ac, pulvinar sit amet odio. Nulla vitae. "

\textbf{à été chiffré : }[2984997306659616885403234044691507117892, 524239168200635302096730407294077786241, 
641427556272420534589639353268340765986, 2451047400590571846880830778373999556399,
......
3327835986657716232182250653126412308245, 1939631357461029317567847319438648758274,
880062521405306667062101246208064289226, 955451065725524141511404637428216027204]

    \end{Verbatim}
\end{code}

\section{\texorpdfstring{\textbf{Question
3}}{Question 3}}\label{question-3}

\subsection{\texorpdfstring{\textbf{Déchiffrement RSA d'un
message}}{Déchiffrement RSA d'un message}}\label{duxe9chiffrement-rsa-dun-message}

\subsubsection{\texorpdfstring{\textbf{Définition}}{Définition}}\label{duxe9finition}

   \begin{code}\begin{Verbatim}[commandchars=\\\{\}]
{\color{incolor}In [{\color{incolor}0}]:} \PY{k}{def} \PY{n+nf}{rsaToMessage}\PY{p}{(}\PY{n}{rsa}\PY{p}{,} \PY{n}{n}\PY{p}{,} \PY{n}{a}\PY{p}{)}\PY{p}{:}
        
          \PY{n}{result} \PY{o}{=} \PY{p}{[}\PY{p}{]}
          \PY{k}{for} \PY{n}{elt} \PY{o+ow}{in} \PY{n}{rsa}\PY{p}{:}
            \PY{n}{result}\PY{o}{.}\PY{n}{append}\PY{p}{(}\PY{n}{chiffrementRSA}\PY{p}{(}\PY{n}{elt}\PY{p}{,} \PY{n}{n}\PY{p}{,} \PY{n}{a}\PY{p}{)}\PY{p}{)}
          
          \PY{n}{taille} \PY{o}{=} \PY{n}{tailleBlocMax}\PY{p}{(}\PY{n}{n}\PY{p}{)}
          \PY{n}{l} \PY{o}{=} \PY{n}{decodage}\PY{p}{(}\PY{n}{result}\PY{p}{,} \PY{n}{tailleBloc}\PY{o}{=}\PY{n}{taille}\PY{p}{)}
        
          \PY{k}{return} \PY{n}{l}
\end{Verbatim}
\end{code}


\subsubsection{\texorpdfstring{\textbf{Test}}{Test}}\label{test}

Le message est donc bien déchiffré, exepté, encore une fois, pour le
caractère spécial "à" qui a été remplacé par un "@".

   \begin{code}\begin{Verbatim}[commandchars=\\\{\}]
{\color{incolor}In [{\color{incolor}215}]:} \PY{n+nb}{print}\PY{p}{(}\PY{l+s+s2}{\PYZdq{}}\PY{l+s+s2}{Le message à déchiffrer est :}\PY{l+s+s2}{\PYZdq{}}\PY{p}{,}\PY{n}{msgRSA}\PY{p}{)}
          \PY{n+nb}{print}\PY{p}{(}\PY{l+s+s2}{\PYZdq{}}\PY{l+s+s2}{\PYZdq{}}\PY{p}{)}
          
          \PY{n}{msgAfter} \PY{o}{=} \PY{n}{rsaToMessage}\PY{p}{(}\PY{n}{msgRSA}\PY{p}{,} \PY{n}{n}\PY{p}{,} \PY{n}{a}\PY{p}{)}
          \PY{n+nb}{print}\PY{p}{(}\PY{l+s+s2}{\PYZdq{}}\PY{l+s+s2}{Message déchiffré :}\PY{l+s+s2}{\PYZdq{}}\PY{p}{,}\PY{n}{msgAfter}\PY{p}{)}
          \PY{n+nb}{print}\PY{p}{(}\PY{l+s+s2}{\PYZdq{}}\PY{l+s+s2}{     Vérification :}\PY{l+s+s2}{\PYZdq{}}\PY{p}{,}\PY{n}{msg}\PY{p}{)}
\end{Verbatim}
\end{code}


   \begin{code}\begin{Verbatim}[commandchars=\\\{\}]
\textbf{Le message à déchiffrer est : }
[2984997306659616885403234044691507117892, 524239168200635302096730407294077786241, 
641427556272420534589639353268340765986, 2451047400590571846880830778373999556399,
......
3327835986657716232182250653126412308245, 1939631357461029317567847319438648758274,
880062521405306667062101246208064289226, 955451065725524141511404637428216027204]

\textbf{Message déchiffré : }@ Lorem ipsum dolor sit amet, consectetur adipiscing elit. Quisque egestas eget  
......
tristique at. Morbi augue purus, elementum non posuere ac, pulvinar sit amet odio. Nulla vitae.

     \textbf{Vérification : }@ Lorem ipsum dolor sit amet, consectetur adipiscing elit. Quisque egestas eget 
......
tristique at. Morbi augue purus, elementum non posuere ac, pulvinar sit amet odio. Nulla vitae.

    \end{Verbatim}
\end{code}

\section{\texorpdfstring{\textbf{Question
4}}{Question 4}}\label{question-4}

\subsection{\texorpdfstring{\textbf{Déchiffrement
frauduleux}}{Déchiffrement frauduleux}}\label{duxe9chiffrement-frauduleux}

\begin{enumerate}
\def\labelenumi{\arabic{enumi})}
\tightlist
\item
  \(x\) doit être premier avec \(n\) :
\end{enumerate}

Le principe du chiffrement RSA est fondé sur la complexité de la
décomposition d'un grand nombre en facteurs premiers, en d'autres mots,
de retrouver \(p\) et \(q\) à partir de \(n\).

Si \(x\) n'est pas premier avec \(n\), alors \(x\) vaut soit \(q\), soit
p (\(x\) étant inférieur à \(n\)), ainsi si \(x\) n'est pas premier, il
est possible de retrouver les facteurs premiers de n qui sont donc \(x\)
et \(\frac{n}{x}\).

\begin{enumerate}
\def\labelenumi{\arabic{enumi})}
\setcounter{enumi}{1}
\end{enumerate}

\paragraph{}\label{section}

\subsubsection{\texorpdfstring{\textbf{Définition}}{Définition}}\label{duxe9finition}

   \begin{code}\begin{Verbatim}[commandchars=\\\{\}]
{\color{incolor}In [{\color{incolor}0}]:} \PY{k}{def} \PY{n+nf}{probXPrimeWithN}\PY{p}{(}\PY{n}{n}\PY{p}{)}\PY{p}{:}
          
          \PY{n}{xMax} \PY{o}{=} \PY{n}{tailleBlocMax}\PY{p}{(}\PY{n}{n}\PY{p}{)}
          \PY{k}{return} \PY{l+s+s2}{\PYZdq{}}\PY{l+s+s2}{2/}\PY{l+s+s2}{\PYZdq{}}\PY{o}{+}\PY{n+nb}{str}\PY{p}{(}\PY{n+nb}{len}\PY{p}{(}\PY{n}{abc}\PY{p}{)}\PY{o}{*}\PY{o}{*}\PY{n}{xMax}\PY{p}{)}
\end{Verbatim}
\end{code}


\subsubsection{\texorpdfstring{\textbf{Test}}{Test}}\label{test}

   \begin{code}\begin{Verbatim}[commandchars=\\\{\}]
{\color{incolor}In [{\color{incolor}217}]:} \PY{n+nb}{print}\PY{p}{(}\PY{l+s+s2}{\PYZdq{}}\PY{l+s+s2}{La probabilité que le message encodé ne soit pas premier avec n est de :}\PY{l+s+s2}{\PYZdq{}}\PY{p}{,}
                 \PY{n}{probXPrimeWithN}\PY{p}{(}\PY{n}{n}\PY{p}{)}\PY{p}{)}
\end{Verbatim}
\end{code}


   \begin{code}\begin{Verbatim}[commandchars=\\\{\}]
La probabilité que le message encodé ne soit pas premier avec n est de :
2/188919613181312032574569023867244773376

    \end{Verbatim}
\end{code}

\chapter{\texorpdfstring{\textbf{Partie 4 - Génération de nombres
premiers}}{Partie 4 - Génération de nombres premiers}}\label{partie-4---guxe9nuxe9ration-de-nombres-premiers}

\section{\texorpdfstring{\textbf{Question
1}}{Question 1}}\label{question-1}

\subsection{\textbf{Resolution de l'équation \((n-1) =
2^{s}\times d\)}}\label{resolution-de-luxe9quation-n-1-2sd}

\subsubsection{\texorpdfstring{\textbf{Définition}}{Définition}}\label{duxe9finition}

   \begin{code}\begin{Verbatim}[commandchars=\\\{\}]
{\color{incolor}In [{\color{incolor}0}]:} \PY{k}{def} \PY{n+nf}{resolveMillerFormula}\PY{p}{(}\PY{n}{n}\PY{p}{)}\PY{p}{:}
          \PY{k}{assert} \PY{n}{n}\PY{o}{\PYZpc{}}\PY{k}{2}, \PYZdq{}n doit être impair\PYZdq{}
        
          \PY{n}{s} \PY{o}{=} \PY{l+m+mi}{0}
          
          \PY{n}{nMinus1Bin} \PY{o}{=} \PY{n+nb}{bin}\PY{p}{(}\PY{n}{n}\PY{o}{\PYZhy{}}\PY{l+m+mi}{1}\PY{p}{)}\PY{p}{[}\PY{l+m+mi}{2}\PY{p}{:}\PY{p}{]}
          \PY{k}{while} \PY{p}{(}\PY{n}{s}\PY{o}{\PYZlt{}}\PY{n+nb}{len}\PY{p}{(}\PY{n}{nMinus1Bin}\PY{p}{)} \PY{o+ow}{and} \PY{n}{nMinus1Bin}\PY{p}{[}\PY{n+nb}{len}\PY{p}{(}\PY{n}{nMinus1Bin}\PY{p}{)}\PY{o}{\PYZhy{}}\PY{l+m+mi}{1}\PY{o}{\PYZhy{}}\PY{n}{s}\PY{p}{]}\PY{o}{==}\PY{l+s+s1}{\PYZsq{}}\PY{l+s+s1}{0}\PY{l+s+s1}{\PYZsq{}}\PY{p}{)}\PY{p}{:}
            \PY{n}{s}\PY{o}{+}\PY{o}{=}\PY{l+m+mi}{1}
        
          \PY{n}{d} \PY{o}{=} \PY{n+nb}{int}\PY{p}{(}\PY{n}{nMinus1Bin}\PY{p}{[}\PY{p}{:}\PY{n+nb}{len}\PY{p}{(}\PY{n}{nMinus1Bin}\PY{p}{)}\PY{o}{\PYZhy{}}\PY{n}{s}\PY{p}{]}\PY{p}{,} \PY{l+m+mi}{2}\PY{p}{)}
          
          \PY{k}{return} \PY{n}{s}\PY{p}{,} \PY{n}{d}
\end{Verbatim}
\end{code}


\subsubsection{\texorpdfstring{\textbf{Test}}{Test}}\label{test}

   \begin{code}\begin{Verbatim}[commandchars=\\\{\}]
{\color{incolor}In [{\color{incolor}219}]:} \PY{n}{n} \PY{o}{=} \PY{l+m+mi}{565}
          
          \PY{n}{s}\PY{p}{,} \PY{n}{d} \PY{o}{=} \PY{n}{resolveMillerFormula}\PY{p}{(}\PY{n}{n}\PY{p}{)}
          \PY{n+nb}{print}\PY{p}{(}\PY{n}{n}\PY{o}{\PYZhy{}}\PY{l+m+mi}{1}\PY{p}{,} \PY{l+s+s2}{\PYZdq{}}\PY{l+s+s2}{= 2 \PYZca{}}\PY{l+s+s2}{\PYZdq{}}\PY{p}{,} \PY{n}{s}\PY{p}{,} \PY{l+s+s2}{\PYZdq{}}\PY{l+s+s2}{*}\PY{l+s+s2}{\PYZdq{}}\PY{p}{,} \PY{n}{d}\PY{p}{)}
\end{Verbatim}
\end{code}


   \begin{code}\begin{Verbatim}[commandchars=\\\{\}]
564 = 2 \^{} 2 * 141

    \end{Verbatim}
\end{code}

\subsection{\texorpdfstring{\textbf{Test de Miller en base
définie}}{Test de Miller en base définie}}\label{test-de-miller-en-base-duxe9finie}

\subsubsection{\texorpdfstring{\textbf{Définition}}{Définition}}\label{duxe9finition}

   \begin{code}\begin{Verbatim}[commandchars=\\\{\}]
{\color{incolor}In [{\color{incolor}0}]:} \PY{k}{def} \PY{n+nf}{testMillerBase}\PY{p}{(}\PY{n}{n}\PY{p}{,} \PY{n}{a}\PY{p}{)}\PY{p}{:}
          \PY{k}{assert} \PY{n}{n}\PY{o}{\PYZgt{}}\PY{l+m+mi}{3}\PY{p}{,} \PY{l+s+s2}{\PYZdq{}}\PY{l+s+s2}{n doit être supérieur à 3}\PY{l+s+s2}{\PYZdq{}}
          \PY{k}{assert} \PY{n}{a}\PY{o}{\PYZlt{}}\PY{n}{n}\PY{p}{,} \PY{l+s+s2}{\PYZdq{}}\PY{l+s+s2}{a doit être inférieur à n}\PY{l+s+s2}{\PYZdq{}}
          
          \PY{n}{s}\PY{p}{,} \PY{n}{d} \PY{o}{=} \PY{n}{resolveMillerFormula}\PY{p}{(}\PY{n}{n}\PY{p}{)}
        
          \PY{n}{x} \PY{o}{=} \PY{n}{expoRapide}\PY{p}{(}\PY{n}{a}\PY{p}{,} \PY{n}{d}\PY{p}{,} \PY{n}{n}\PY{p}{)}
        
          \PY{k}{if} \PY{n}{x}\PY{o}{==}\PY{l+m+mi}{1} \PY{o+ow}{or} \PY{n}{x}\PY{o}{==}\PY{n}{n}\PY{o}{\PYZhy{}}\PY{l+m+mi}{1}\PY{p}{:}
            \PY{k}{return} \PY{k+kc}{False}
        
          \PY{k}{for} \PY{n}{\PYZus{}} \PY{o+ow}{in} \PY{n+nb}{range}\PY{p}{(}\PY{n}{s}\PY{o}{\PYZhy{}}\PY{l+m+mi}{1}\PY{p}{)}\PY{p}{:}
            \PY{n}{x} \PY{o}{=} \PY{n}{expoRapide}\PY{p}{(}\PY{n}{x}\PY{p}{,} \PY{n}{x}\PY{p}{,} \PY{n}{n}\PY{p}{)}
            \PY{k}{if} \PY{n}{x}\PY{o}{==}\PY{n}{n}\PY{o}{\PYZhy{}}\PY{l+m+mi}{1}\PY{p}{:}
              \PY{k}{return} \PY{k+kc}{False}
          
          \PY{k}{return} \PY{k+kc}{True}
\end{Verbatim}
\end{code}


\subsubsection{\texorpdfstring{\textbf{Test}}{Test}}\label{test}

   \begin{code}\begin{Verbatim}[commandchars=\\\{\}]
{\color{incolor}In [{\color{incolor}221}]:} \PY{n}{n} \PY{o}{=} \PY{l+m+mi}{23}
          \PY{n}{a} \PY{o}{=} \PY{l+m+mi}{6}
          \PY{n}{r} \PY{o}{=} \PY{n}{testMillerBase}\PY{p}{(}\PY{n}{n}\PY{p}{,} \PY{n}{a}\PY{p}{)}
          \PY{n+nb}{print}\PY{p}{(}\PY{n}{n}\PY{p}{,}\PY{l+s+s2}{\PYZdq{}}\PY{l+s+s2}{est premier :}\PY{l+s+s2}{\PYZdq{}}\PY{p}{,}\PY{o+ow}{not} \PY{n}{r}\PY{p}{)}
          \PY{n+nb}{print}\PY{p}{(}\PY{l+s+s2}{\PYZdq{}}\PY{l+s+s2}{Comparaison avec la bibliothèque Sympy :}\PY{l+s+s2}{\PYZdq{}}\PY{p}{,} \PY{n}{sy}\PY{o}{.}\PY{n}{isprime}\PY{p}{(}\PY{n}{n}\PY{p}{)}\PY{p}{)}
          \PY{n+nb}{print}\PY{p}{(}\PY{l+s+s2}{\PYZdq{}}\PY{l+s+s2}{\PYZdq{}}\PY{p}{)}
          
          \PY{n}{n} \PY{o}{=} \PY{l+m+mi}{561}
          \PY{n}{a} \PY{o}{=} \PY{l+m+mi}{50}
          \PY{n}{r} \PY{o}{=} \PY{n}{testMillerBase}\PY{p}{(}\PY{n}{n}\PY{p}{,} \PY{n}{a}\PY{p}{)}
          \PY{n+nb}{print}\PY{p}{(}\PY{n}{n}\PY{p}{,}\PY{l+s+s2}{\PYZdq{}}\PY{l+s+s2}{est premier :}\PY{l+s+s2}{\PYZdq{}}\PY{p}{,}\PY{o+ow}{not} \PY{n}{r}\PY{p}{)}
          \PY{n+nb}{print}\PY{p}{(}\PY{l+s+s2}{\PYZdq{}}\PY{l+s+s2}{Comparaison avec la bibliothèque Sympy :}\PY{l+s+s2}{\PYZdq{}}\PY{p}{,} \PY{n}{sy}\PY{o}{.}\PY{n}{isprime}\PY{p}{(}\PY{n}{n}\PY{p}{)}\PY{p}{)}
          \PY{n+nb}{print}\PY{p}{(}\PY{l+s+s2}{\PYZdq{}}\PY{l+s+s2}{Le témoin 50 est un menteur pour l}\PY{l+s+s2}{\PYZsq{}}\PY{l+s+s2}{entier 561.}\PY{l+s+s2}{\PYZdq{}}\PY{p}{)}
\end{Verbatim}
\end{code}


   \begin{code}\begin{Verbatim}[commandchars=\\\{\}]
23 est premier : True
Comparaison avec la bibliothèque Sympy : True

561 est premier : True
Comparaison avec la bibliothèque Sympy : False
Le témoin 50 est un menteur pour l'entier 561.

    \end{Verbatim}
\end{code}

\section{\texorpdfstring{\textbf{Question
2}}{Question 2}}\label{question-2}

\subsection{\texorpdfstring{\textbf{Test de
Miller}}{Test de Miller}}\label{test-de-miller}

\subsubsection{\texorpdfstring{\textbf{Définition}}{Définition}}\label{duxe9finition}

   \begin{code}\begin{Verbatim}[commandchars=\\\{\}]
{\color{incolor}In [{\color{incolor}0}]:} \PY{k}{def} \PY{n+nf}{testMiller}\PY{p}{(}\PY{n}{n}\PY{p}{,} \PY{n}{m}\PY{o}{=}\PY{l+m+mi}{20}\PY{p}{)}\PY{p}{:}
          \PY{k}{if} \PY{n}{n}\PY{o}{\PYZlt{}}\PY{o}{=}\PY{l+m+mi}{1}\PY{p}{:}
            \PY{k}{return} \PY{k+kc}{False}
          \PY{k}{if} \PY{n}{n}\PY{o}{\PYZlt{}}\PY{o}{=}\PY{l+m+mi}{3}\PY{p}{:}
            \PY{k}{return} \PY{k+kc}{True}
          \PY{k}{if} \PY{o+ow}{not} \PY{n}{n}\PY{o}{\PYZpc{}}\PY{k}{2}:
            \PY{k}{return} \PY{k+kc}{False}
          
          \PY{k}{for} \PY{n}{\PYZus{}} \PY{o+ow}{in} \PY{n+nb}{range}\PY{p}{(}\PY{n}{m}\PY{p}{)}\PY{p}{:}
            \PY{n}{a} \PY{o}{=} \PY{n}{rand}\PY{o}{.}\PY{n}{randint}\PY{p}{(}\PY{l+m+mi}{2}\PY{p}{,}\PY{n}{n}\PY{o}{\PYZhy{}}\PY{l+m+mi}{2}\PY{p}{)}
            \PY{k}{if} \PY{n}{testMillerBase}\PY{p}{(}\PY{n}{n}\PY{p}{,} \PY{n}{a}\PY{p}{)}\PY{p}{:}
              \PY{k}{return} \PY{k+kc}{False}
          \PY{k}{return} \PY{k+kc}{True}
\end{Verbatim}
\end{code}


\subsubsection{\texorpdfstring{\textbf{Test}}{Test}}\label{test}

   \begin{code}\begin{Verbatim}[commandchars=\\\{\}]
{\color{incolor}In [{\color{incolor}223}]:} \PY{n}{n} \PY{o}{=} \PY{l+m+mi}{23}
          \PY{n}{r} \PY{o}{=} \PY{n}{testMiller}\PY{p}{(}\PY{n}{n}\PY{p}{)}
          \PY{n+nb}{print}\PY{p}{(}\PY{n}{n}\PY{p}{,}\PY{l+s+s2}{\PYZdq{}}\PY{l+s+s2}{est premier :}\PY{l+s+s2}{\PYZdq{}}\PY{p}{,}\PY{n}{r}\PY{p}{)}
          \PY{n+nb}{print}\PY{p}{(}\PY{l+s+s2}{\PYZdq{}}\PY{l+s+s2}{Comparaison avec la bibliothèque Sympy :}\PY{l+s+s2}{\PYZdq{}}\PY{p}{,} \PY{n}{sy}\PY{o}{.}\PY{n}{isprime}\PY{p}{(}\PY{n}{n}\PY{p}{)}\PY{p}{)}
          \PY{n+nb}{print}\PY{p}{(}\PY{l+s+s2}{\PYZdq{}}\PY{l+s+s2}{\PYZdq{}}\PY{p}{)}
          
          \PY{n}{n} \PY{o}{=} \PY{l+m+mi}{221}
          \PY{n}{r} \PY{o}{=} \PY{n}{testMiller}\PY{p}{(}\PY{n}{n}\PY{p}{)}
          \PY{n+nb}{print}\PY{p}{(}\PY{n}{n}\PY{p}{,}\PY{l+s+s2}{\PYZdq{}}\PY{l+s+s2}{est premier :}\PY{l+s+s2}{\PYZdq{}}\PY{p}{,}\PY{n}{r}\PY{p}{)}
          \PY{n+nb}{print}\PY{p}{(}\PY{l+s+s2}{\PYZdq{}}\PY{l+s+s2}{Comparaison avec la bibliothèque Sympy :}\PY{l+s+s2}{\PYZdq{}}\PY{p}{,} \PY{n}{sy}\PY{o}{.}\PY{n}{isprime}\PY{p}{(}\PY{n}{n}\PY{p}{)}\PY{p}{)}
          \PY{n+nb}{print}\PY{p}{(}\PY{l+s+s2}{\PYZdq{}}\PY{l+s+s2}{L}\PY{l+s+s2}{\PYZsq{}}\PY{l+s+s2}{utilisation de multiples témoins, permet de mettre en défaut les témoins}
                 \PY{l+s+s2}{menteurs}\PY{l+s+s2}{\PYZsq{}}\PY{l+s+s2}{\PYZdq{}}\PY{p}{)}
\end{Verbatim}
\end{code}


   \begin{code}\begin{Verbatim}[commandchars=\\\{\}]
23 est premier : True
Comparaison avec la bibliothèque Sympy : True

221 est premier : False
Comparaison avec la bibliothèque Sympy : False
L'utilisation de multiples témoins, permet de mettre en défaut les témoins menteurs'

    \end{Verbatim}
\end{code}

\section{\texorpdfstring{\textbf{Question
3}}{Question 3}}\label{question-3}

\subsection{\texorpdfstring{\textbf{Générateur de nombre
premier}}{Générateur de nombre premier}}\label{guxe9nuxe9rateur-de-nombre-premier}

\subsubsection{\texorpdfstring{\textbf{Définition}}{Définition}}\label{duxe9finition}

   \begin{code}\begin{Verbatim}[commandchars=\\\{\}]
{\color{incolor}In [{\color{incolor}0}]:} \PY{k}{def} \PY{n+nf}{generateurPremier}\PY{p}{(}\PY{n}{k}\PY{p}{)}\PY{p}{:}
          \PY{k}{assert} \PY{n}{k}\PY{o}{\PYZgt{}}\PY{o}{=}\PY{l+m+mi}{2}\PY{p}{,} \PY{l+s+s2}{\PYZdq{}}\PY{l+s+s2}{k doit être supérieur strictement à 1}\PY{l+s+s2}{\PYZdq{}}
          
          \PY{n}{x} \PY{o}{=} \PY{l+m+mi}{1}
          \PY{k}{while} \PY{p}{(}\PY{o+ow}{not} \PY{n}{testMiller}\PY{p}{(}\PY{n}{x}\PY{p}{,} \PY{l+m+mi}{100}\PY{p}{)}\PY{p}{)}\PY{p}{:}
            \PY{n}{x} \PY{o}{=} \PY{n}{rand}\PY{o}{.}\PY{n}{randint}\PY{p}{(}\PY{l+m+mi}{2}\PY{o}{*}\PY{o}{*}\PY{p}{(}\PY{n}{k}\PY{o}{\PYZhy{}}\PY{l+m+mi}{1}\PY{p}{)}\PY{p}{,} \PY{l+m+mi}{2}\PY{o}{*}\PY{o}{*}\PY{p}{(}\PY{n}{k}\PY{p}{)}\PY{o}{\PYZhy{}}\PY{l+m+mi}{1}\PY{p}{)} \PY{o}{|} \PY{l+m+mi}{1} \PY{c+c1}{\PYZsh{} Permet d'avoir un nombre impair}
        
          \PY{k}{return} \PY{n}{x}
\end{Verbatim}
\end{code}


\subsubsection{\texorpdfstring{\textbf{Test}}{Test}}\label{test}

   \begin{code}\begin{Verbatim}[commandchars=\\\{\}]
{\color{incolor}In [{\color{incolor}225}]:} \PY{n}{power} \PY{o}{=} \PY{p}{[}\PY{l+m+mi}{5}\PY{p}{,} \PY{l+m+mi}{10}\PY{p}{,} \PY{l+m+mi}{100}\PY{p}{,} \PY{l+m+mi}{250}\PY{p}{,} \PY{l+m+mi}{500}\PY{p}{]}\PY{c+c1}{\PYZsh{}, 1000, 1000, 2000, 3000, 4000, 5000]}
          
          \PY{k}{for} \PY{n}{k} \PY{o+ow}{in} \PY{n}{power}\PY{p}{:}
            \PY{n}{t} \PY{o}{=} \PY{n}{micro}\PY{p}{(}\PY{p}{)}
            \PY{n}{x} \PY{o}{=} \PY{n}{generateurPremier}\PY{p}{(}\PY{n}{k}\PY{p}{)}
            \PY{n+nb}{print}\PY{p}{(}\PY{n}{k}\PY{p}{,} \PY{l+s+s2}{\PYZdq{}}\PY{l+s+s2}{\PYZhy{}}\PY{l+s+s2}{\PYZdq{}}\PY{p}{,} \PY{n}{x}\PY{p}{,} \PY{l+s+s2}{\PYZdq{}}\PY{l+s+s2}{est bien premier :}\PY{l+s+s2}{\PYZdq{}}\PY{p}{,}\PY{n}{sy}\PY{o}{.}\PY{n}{isprime}\PY{p}{(}\PY{n}{x}\PY{p}{)}\PY{p}{)}
            \PY{n+nb}{print}\PY{p}{(}\PY{l+s+s2}{\PYZdq{}}\PY{l+s+s2}{\PYZsh{}Nombre premier trouvé en}\PY{l+s+s2}{\PYZdq{}}\PY{p}{,}\PY{n}{displayPeriod}\PY{p}{(}\PY{n}{micro}\PY{p}{(}\PY{p}{)}\PY{o}{\PYZhy{}}\PY{n}{t}\PY{p}{)}\PY{p}{)}
            \PY{n+nb}{print}\PY{p}{(}\PY{l+s+s2}{\PYZdq{}}\PY{l+s+s2}{\PYZdq{}}\PY{p}{)}
\end{Verbatim}
\end{code}


   \begin{code}\begin{Verbatim}[commandchars=\\\{\}]
5 - 23 est bien premier : True
\#Nombre premier trouvé en 1.473ms

10 - 983 est bien premier : True
\#Nombre premier trouvé en 2.412ms

100 - 1115478283892045139697472002619 est bien premier : True
\#Nombre premier trouvé en 32.81ms

250 - 150591927924164123781108730185339606251824 ...... 76530187998579707 est bien premier : True
\#Nombre premier trouvé en 154.745ms

500 - 297550349025194451582781880445926863926293 ...... 06617779917107783 est bien premier : True
\#Nombre premier trouvé en 759.682ms


    \end{Verbatim}
\end{code}

\chapter{\texorpdfstring{\textbf{Partie 5 -
ElGamal}}{Partie 5 - ElGamal}}\label{partie-5---elgamal}

\section{\texorpdfstring{\textbf{Construction de
p}}{Construction de p}}\label{construction-de-p}

\subsection{\texorpdfstring{\textbf{Question
1}}{Question 1}}\label{question-1}

\subsubsection{\texorpdfstring{\textbf{Condition de
complexité}}{Condition de complexité}}\label{condition-de-complexituxe9}

L'algoritme ElGamal repose sur la complexité du problème de logarithme
discret (déchiffrement par force brut), contrairement à
l'exponentiation.

Le problème de logarithme discret requiert une clé \(p\), qui doit être
très grande et nombre premier, et \(p-1\) avec un très grand facteur
premier, d'où \(p = q\times2-1\).

\subsection{\texorpdfstring{\textbf{Question
2}}{Question 2}}\label{question-2}

\subsubsection{\texorpdfstring{\textbf{Génération d'une nombre premier
ElGamal}}{Génération d'une nombre premier ElGamal}}\label{guxe9nuxe9ration-dune-nombre-premier-elgamal}

\paragraph{\texorpdfstring{\textbf{Définition}}{Définition}}\label{duxe9finition}

   \begin{code}\begin{Verbatim}[commandchars=\\\{\}]
{\color{incolor}In [{\color{incolor}0}]:} \PY{k}{def} \PY{n+nf}{premierElGamal}\PY{p}{(}\PY{n}{k}\PY{p}{)}\PY{p}{:}
        
          \PY{n}{q} \PY{o}{=} \PY{l+m+mi}{0}
          \PY{n}{p} \PY{o}{=} \PY{n}{q}\PY{o}{*}\PY{l+m+mi}{2}\PY{o}{+}\PY{l+m+mi}{1}
          \PY{k}{while} \PY{o+ow}{not} \PY{n}{testMiller}\PY{p}{(}\PY{n}{p}\PY{p}{)}\PY{p}{:}
            \PY{n}{q} \PY{o}{=} \PY{n}{generateurPremier}\PY{p}{(}\PY{n}{k}\PY{p}{)}
            \PY{n}{p} \PY{o}{=} \PY{n}{q}\PY{o}{*}\PY{l+m+mi}{2}\PY{o}{+}\PY{l+m+mi}{1}
        
          \PY{k}{return} \PY{n}{p}\PY{p}{,} \PY{n}{q}
\end{Verbatim}
\end{code}


\paragraph{\texorpdfstring{\textbf{Test}}{Test}}\label{test}

   \begin{code}\begin{Verbatim}[commandchars=\\\{\}]
{\color{incolor}In [{\color{incolor}227}]:} \PY{n}{power} \PY{o}{=} \PY{p}{[}\PY{l+m+mi}{2}\PY{p}{,} \PY{l+m+mi}{5}\PY{p}{,} \PY{l+m+mi}{10}\PY{p}{,} \PY{l+m+mi}{100}\PY{p}{]}
          
          \PY{k}{for} \PY{n}{k} \PY{o+ow}{in} \PY{n}{power}\PY{p}{[}\PY{l+m+mi}{1}\PY{p}{:}\PY{p}{]}\PY{p}{:}
            \PY{n}{t} \PY{o}{=} \PY{n}{micro}\PY{p}{(}\PY{p}{)}
            \PY{n}{p}\PY{p}{,} \PY{n}{q} \PY{o}{=} \PY{n}{premierElGamal}\PY{p}{(}\PY{n}{k}\PY{p}{)}
            \PY{n+nb}{print}\PY{p}{(}\PY{n}{k}\PY{p}{,} \PY{l+s+s2}{\PYZdq{}}\PY{l+s+s2}{donne}\PY{l+s+s2}{\PYZdq{}}\PY{p}{,} \PY{n}{p}\PY{p}{,}\PY{l+s+s2}{\PYZdq{}}\PY{l+s+s2}{=}\PY{l+s+s2}{\PYZdq{}}\PY{p}{,} \PY{n}{q}\PY{p}{,}\PY{l+s+s2}{\PYZdq{}}\PY{l+s+s2}{* 2 + 1}\PY{l+s+s2}{\PYZdq{}}\PY{p}{)}
            \PY{n+nb}{print}\PY{p}{(}\PY{n}{p}\PY{p}{,}\PY{l+s+s2}{\PYZdq{}}\PY{l+s+s2}{est premier :}\PY{l+s+s2}{\PYZdq{}}\PY{p}{,}\PY{n}{sy}\PY{o}{.}\PY{n}{isprime}\PY{p}{(}\PY{n}{x}\PY{p}{)}\PY{p}{)}
            \PY{n+nb}{print}\PY{p}{(}\PY{l+s+s2}{\PYZdq{}}\PY{l+s+s2}{\PYZsh{}Nombre premier de ElGamal trouvé en}\PY{l+s+s2}{\PYZdq{}}\PY{p}{,}\PY{n}{displayPeriod}\PY{p}{(}\PY{n}{micro}\PY{p}{(}\PY{p}{)}\PY{o}{\PYZhy{}}\PY{n}{t}\PY{p}{)}\PY{p}{)}
            \PY{n+nb}{print}\PY{p}{(}\PY{l+s+s2}{\PYZdq{}}\PY{l+s+s2}{\PYZdq{}}\PY{p}{)}
\end{Verbatim}
\end{code}


   \begin{code}\begin{Verbatim}[commandchars=\\\{\}]
5 donne 47 = 23 * 2 + 1
47 est premier : True
\#Nombre premier de ElGamal trouvé en 5.809ms

10 donne 1319 = 659 * 2 + 1
1319 est premier : True
\#Nombre premier de ElGamal trouvé en 19.484ms

100 donne 1309176998925936328789683684599 = 654588499462968164394841842299 * 2 + 1
1309176998925936328789683684599 est premier : True
\#Nombre premier de ElGamal trouvé en 169.448ms
    \end{Verbatim}
\end{code}

\section{\texorpdfstring{\textbf{Construction d'une clé
(p,a)}}{Construction d'une clé (p,a)}}\label{construction-dune-cluxe9-pa}

\subsection{\texorpdfstring{\textbf{Question
1}}{Question 1}}\label{question-1}

\titleformat{\subsubsection}{}{}{}{}
\subsubsection{\texorpdfstring{\textbf{Clé Privé
ElGamal}}{Clé Privé ElGamal}}\label{cluxe9-privuxe9-elgamal}

\paragraph{\texorpdfstring{\textbf{Définition}}{Définition}}\label{duxe9finition}

   \begin{code}\begin{Verbatim}[commandchars=\\\{\}]
{\color{incolor}In [{\color{incolor}0}]:} \PY{k}{def} \PY{n+nf}{clePriveElGamal}\PY{p}{(}\PY{n}{k}\PY{p}{)}\PY{p}{:}
        
          \PY{n}{p}\PY{p}{,} \PY{n}{\PYZus{}} \PY{o}{=} \PY{n}{premierElGamal}\PY{p}{(}\PY{n}{k}\PY{p}{)}
          \PY{n}{a} \PY{o}{=} \PY{n}{rand}\PY{o}{.}\PY{n}{randint}\PY{p}{(}\PY{l+m+mi}{0}\PY{p}{,} \PY{n}{p}\PY{o}{\PYZhy{}}\PY{l+m+mi}{2}\PY{p}{)}
        
          \PY{k}{return} \PY{n}{p}\PY{p}{,} \PY{n}{a}
\end{Verbatim}
\end{code}


\paragraph{\texorpdfstring{\textbf{Test}}{Test}}\label{test}

   \begin{code}\begin{Verbatim}[commandchars=\\\{\}]
{\color{incolor}In [{\color{incolor}229}]:} \PY{n}{k} \PY{o}{=} \PY{l+m+mi}{100}
          \PY{n}{p}\PY{p}{,} \PY{n}{a} \PY{o}{=} \PY{n}{clePriveElGamal}\PY{p}{(}\PY{n}{k}\PY{p}{)}
          \PY{n+nb}{print}\PY{p}{(}\PY{l+s+s2}{\PYZdq{}}\PY{l+s+s2}{Génération d}\PY{l+s+s2}{\PYZsq{}}\PY{l+s+s2}{une clé privé ElGamal (p,a) :}\PY{l+s+s2}{\PYZdq{}}\PY{p}{,} \PY{p}{(}\PY{n}{p}\PY{p}{,} \PY{n}{a}\PY{p}{)}\PY{p}{)}
\end{Verbatim}
\end{code}


   \begin{code}\begin{Verbatim}[commandchars=\\\{\}]
Génération d'une clé privé ElGamal (p,a) : (2024719205781249178367334290303,
                                            258364839610907258768697440970)

    \end{Verbatim}
\end{code}

\subsubsection{\texorpdfstring{\textbf{Clé Public
ElGamal}}{Clé Public ElGamal}}\label{cluxe9-public-elgamal}

\paragraph{\texorpdfstring{\textbf{Définition}}{Définition}}\label{duxe9finition}

   \begin{code}\begin{Verbatim}[commandchars=\\\{\}]
{\color{incolor}In [{\color{incolor}0}]:} \PY{k}{def} \PY{n+nf}{clePublicElGamal}\PY{p}{(}\PY{n}{p}\PY{p}{,} \PY{n}{a}\PY{p}{)}\PY{p}{:}
          
          \PY{n}{m} \PY{o}{=} \PY{n}{rand}\PY{o}{.}\PY{n}{randint}\PY{p}{(}\PY{l+m+mi}{0}\PY{p}{,} \PY{n}{p}\PY{o}{\PYZhy{}}\PY{l+m+mi}{1}\PY{p}{)}
          \PY{n}{n} \PY{o}{=} \PY{n}{expoRapide}\PY{p}{(}\PY{n}{m}\PY{p}{,} \PY{n}{a}\PY{p}{,} \PY{n}{p}\PY{p}{)}
        
          \PY{k}{return} \PY{n}{p}\PY{p}{,} \PY{n}{m}\PY{p}{,} \PY{n}{n}
\end{Verbatim}
\end{code}


\paragraph{\texorpdfstring{\textbf{Test}}{Test}}\label{test}

   \begin{code}\begin{Verbatim}[commandchars=\\\{\}]
{\color{incolor}In [{\color{incolor}231}]:} \PY{n}{\PYZus{}}\PY{p}{,} \PY{n}{m}\PY{p}{,} \PY{n}{n} \PY{o}{=} \PY{n}{clePublicElGamal}\PY{p}{(}\PY{n}{p}\PY{p}{,} \PY{n}{a}\PY{p}{)}
          \PY{n+nb}{print}\PY{p}{(}\PY{l+s+s2}{\PYZdq{}}\PY{l+s+s2}{Génération d}\PY{l+s+s2}{\PYZsq{}}\PY{l+s+s2}{une clé public ElGamal (p,m,n) :}\PY{l+s+s2}{\PYZdq{}}\PY{p}{,} \PY{p}{(}\PY{n}{p}\PY{p}{,} \PY{n}{m}\PY{p}{,} \PY{n}{n}\PY{p}{)}\PY{p}{)}
\end{Verbatim}
\end{code}


   \begin{code}\begin{Verbatim}[commandchars=\\\{\}]
Génération d'une clé public ElGamal (p,m,n) : (2024719205781249178367334290303,
                                               1364958218230270862059352890533, 
                                               336716890560643709019974134728)

    \end{Verbatim}
\end{code}

\newpage
\titleformat{\subsubsection}{\filcenter}{}{}{}
\section{\texorpdfstring{\textbf{Implémentation de
ElGamal}}{Implémentation de ElGamal}}\label{impluxe9mentation-de-elgamal}

\subsection{\texorpdfstring{\textbf{Chiffrement}}{Chiffrement}}\label{chiffrement}
\subsubsection{\texorpdfstring{\textbf{Définition}}{Définition}}\label{duxe9finition}

   \begin{code}\begin{Verbatim}[commandchars=\\\{\}]
{\color{incolor}In [{\color{incolor}0}]:} \PY{k}{def} \PY{n+nf}{chiffrementElGamal}\PY{p}{(}\PY{n}{x}\PY{p}{,} \PY{n}{public}\PY{p}{)}\PY{p}{:}
        
          \PY{n}{p}\PY{p}{,} \PY{n}{m}\PY{p}{,} \PY{n}{n} \PY{o}{=} \PY{n}{public}
          \PY{k}{assert} \PY{n}{x}\PY{o}{\PYZlt{}}\PY{n}{p}\PY{p}{,} \PY{l+s+s2}{\PYZdq{}}\PY{l+s+s2}{x doit être inférieur à p}\PY{l+s+s2}{\PYZdq{}}
        
          \PY{n}{k} \PY{o}{=} \PY{n}{rand}\PY{o}{.}\PY{n}{randint}\PY{p}{(}\PY{l+m+mi}{0}\PY{p}{,} \PY{n}{p}\PY{o}{\PYZhy{}}\PY{l+m+mi}{1}\PY{p}{)}
        
          \PY{n}{y1} \PY{o}{=} \PY{n}{expoRapide}\PY{p}{(}\PY{n}{m}\PY{p}{,} \PY{n}{k}\PY{p}{,} \PY{n}{p}\PY{p}{)}
          \PY{n}{y2} \PY{o}{=} \PY{n}{x}\PY{o}{*}\PY{n}{expoRapide}\PY{p}{(}\PY{n}{n}\PY{p}{,} \PY{n}{k}\PY{p}{,} \PY{n}{p}\PY{p}{)}
        
          \PY{k}{return} \PY{n}{y1}\PY{p}{,} \PY{n}{y2}
\end{Verbatim}
\end{code}


\subsubsection{\texorpdfstring{\textbf{Test}}{Test}}\label{test}

   \begin{code}\begin{Verbatim}[commandchars=\\\{\}]
{\color{incolor}In [{\color{incolor}233}]:} \PY{n}{msg} \PY{o}{=} \PY{l+s+s2}{\PYZdq{}}\PY{l+s+s2}{Hello World !}\PY{l+s+s2}{\PYZdq{}}
          \PY{n+nb}{print}\PY{p}{(}\PY{l+s+s2}{\PYZdq{}}\PY{l+s+s2}{Le message est : }\PY{l+s+se}{\PYZbs{}\PYZdq{}}\PY{l+s+s2}{\PYZdq{}}\PY{o}{+}\PY{n}{msg}\PY{o}{+}\PY{l+s+s2}{\PYZdq{}}\PY{l+s+se}{\PYZbs{}\PYZdq{}}\PY{l+s+s2}{\PYZdq{}}\PY{p}{)}
          \PY{n+nb}{print}\PY{p}{(}\PY{l+s+s2}{\PYZdq{}}\PY{l+s+s2}{\PYZdq{}}\PY{p}{)}
          
          \PY{n+nb}{print}\PY{p}{(}\PY{l+s+s2}{\PYZdq{}}\PY{l+s+s2}{Un bloc est de taille maximum :}\PY{l+s+s2}{\PYZdq{}}\PY{p}{,}\PY{n}{tailleBlocMax}\PY{p}{(}\PY{n}{p}\PY{p}{)}\PY{p}{)}
          \PY{n+nb}{print}\PY{p}{(}\PY{l+s+s2}{\PYZdq{}}\PY{l+s+s2}{\PYZdq{}}\PY{p}{)}
          
          \PY{n}{cMsg} \PY{o}{=} \PY{n}{encodageBloc}\PY{p}{(}\PY{n}{msg}\PY{p}{)}
          
          \PY{n}{elGamal} \PY{o}{=} \PY{n}{chiffrementElGamal}\PY{p}{(}\PY{n}{cMsg}\PY{p}{,} \PY{p}{(}\PY{n}{p}\PY{p}{,} \PY{n}{m}\PY{p}{,} \PY{n}{n}\PY{p}{)}\PY{p}{)}
          \PY{n+nb}{print}\PY{p}{(}\PY{l+s+s2}{\PYZdq{}}\PY{l+s+s2}{Le bloc chiffré donne :}\PY{l+s+s2}{\PYZdq{}}\PY{p}{,} \PY{n}{elGamal}\PY{p}{)}
\end{Verbatim}
\end{code}


   \begin{code}\begin{Verbatim}[commandchars=\\\{\}]
Le message est : "Hello World !"

Un bloc est de taille maximum : 15

Le bloc chiffré donne : (1974869574197710043720909294406,
6697713938789838788459408160074468572429525393096270496)

    \end{Verbatim}
\end{code}

\subsection{\texorpdfstring{\textbf{Déchiffrement}}{Déchiffrement}}\label{duxe9chiffrement}

\subsubsection{\texorpdfstring{\textbf{Définition}}{Définition}}\label{duxe9finition}

   \begin{code}\begin{Verbatim}[commandchars=\\\{\}]
{\color{incolor}In [{\color{incolor}0}]:} \PY{k}{def} \PY{n+nf}{dechiffrementElGamal}\PY{p}{(}\PY{n}{y}\PY{p}{,} \PY{n}{prive}\PY{p}{)}\PY{p}{:}
        
          \PY{n}{y1}\PY{p}{,} \PY{n}{y2} \PY{o}{=} \PY{n}{y}
          \PY{n}{p}\PY{p}{,} \PY{n}{a} \PY{o}{=} \PY{n}{prive}
          \PY{n}{x} \PY{o}{=} \PY{p}{(}\PY{n}{expoRapide}\PY{p}{(}\PY{n}{y1}\PY{p}{,} \PY{n}{p}\PY{o}{\PYZhy{}}\PY{l+m+mi}{1}\PY{o}{\PYZhy{}}\PY{n}{a}\PY{p}{,} \PY{n}{p}\PY{p}{)} \PY{o}{*} \PY{n}{y2}\PY{p}{)} \PY{o}{\PYZpc{}}\PY{k}{p}
        
          \PY{k}{return} \PY{n}{x}
\end{Verbatim}
\end{code}

\newpage
\subsubsection{\texorpdfstring{\textbf{Test}}{Test}}\label{test}

   \begin{code}\begin{Verbatim}[commandchars=\\\{\}]
{\color{incolor}In [{\color{incolor}235}]:} \PY{n+nb}{print}\PY{p}{(}\PY{l+s+s2}{\PYZdq{}}\PY{l+s+s2}{Le message à déchiffré est :}\PY{l+s+s2}{\PYZdq{}}\PY{p}{,} \PY{n}{elGamal}\PY{p}{)}
          \PY{n+nb}{print}\PY{p}{(}\PY{l+s+s2}{\PYZdq{}}\PY{l+s+s2}{\PYZdq{}}\PY{p}{)}
          
          \PY{n}{dMsg} \PY{o}{=} \PY{n}{dechiffrementElGamal}\PY{p}{(}\PY{n}{elGamal}\PY{p}{,} \PY{p}{(}\PY{n}{p}\PY{p}{,} \PY{n}{a}\PY{p}{)}\PY{p}{)}
          \PY{n+nb}{print}\PY{p}{(}\PY{l+s+s2}{\PYZdq{}}\PY{l+s+s2}{Le bloc déchiffré donne :}\PY{l+s+s2}{\PYZdq{}}\PY{p}{,} \PY{n}{dMsg}\PY{p}{)}
          \PY{n+nb}{print}\PY{p}{(}\PY{l+s+s2}{\PYZdq{}}\PY{l+s+s2}{\PYZdq{}}\PY{p}{)}
          
          \PY{n}{newMsg} \PY{o}{=} \PY{n}{decodageBloc}\PY{p}{(}\PY{n}{dMsg}\PY{p}{)}
          \PY{n+nb}{print}\PY{p}{(}\PY{l+s+s2}{\PYZdq{}}\PY{l+s+s2}{Le message déchiffré et décodé donne :}\PY{l+s+s2}{\PYZdq{}}\PY{p}{,}\PY{n}{newMsg}\PY{p}{)}
          \PY{n+nb}{print}\PY{p}{(}\PY{l+s+s2}{\PYZdq{}}\PY{l+s+s2}{                     Message initial :}\PY{l+s+s2}{\PYZdq{}}\PY{p}{,}\PY{n}{msg}\PY{p}{)}
          
          \PY{n}{cMsg} \PY{o}{=} \PY{n}{encodageBloc}\PY{p}{(}\PY{n}{msg}\PY{p}{)}
\end{Verbatim}
\end{code}


   \begin{code}\begin{Verbatim}[commandchars=\\\{\}]
Le message à déchiffré est : (1974869574197710043720909294406,
6697713938789838788459408160074468572429525393096270496)

Le bloc déchiffré donne : 6379322887179225045360026

Le message déchiffré et décodé donne : Hello World !
                     Message initial : Hello World !

    \end{Verbatim}
\end{code}

\subsection{\texorpdfstring{\textbf{Chiffrement ElGamal d'un message
entier}}{Chiffrement ElGamal d'un message entier}}\label{chiffrement-elgamal-dun-message-entier}

\subsubsection{\texorpdfstring{\textbf{Définition}}{Définition}}\label{duxe9finition}

   \begin{code}\begin{Verbatim}[commandchars=\\\{\}]
{\color{incolor}In [{\color{incolor}0}]:} \PY{k}{def} \PY{n+nf}{messageToElGamal}\PY{p}{(}\PY{n}{msg}\PY{p}{,} \PY{n}{public}\PY{p}{)}\PY{p}{:}
        
          \PY{n}{p}\PY{p}{,} \PY{n}{\PYZus{}}\PY{p}{,} \PY{n}{\PYZus{}} \PY{o}{=} \PY{n}{public}
        
          \PY{n}{taille} \PY{o}{=} \PY{n}{tailleBlocMax}\PY{p}{(}\PY{n}{p}\PY{p}{)}
          \PY{n}{l} \PY{o}{=} \PY{n}{encodage}\PY{p}{(}\PY{n}{msg}\PY{p}{,} \PY{n}{tailleBloc}\PY{o}{=}\PY{n}{taille}\PY{p}{)}
        
          \PY{n}{result} \PY{o}{=} \PY{p}{[}\PY{p}{]}
          \PY{k}{for} \PY{n}{elt} \PY{o+ow}{in} \PY{n}{l}\PY{p}{:}
            \PY{n}{result}\PY{o}{.}\PY{n}{append}\PY{p}{(}\PY{n}{chiffrementElGamal}\PY{p}{(}\PY{n}{elt}\PY{p}{,} \PY{n}{public}\PY{p}{)}\PY{p}{)}
          
          \PY{k}{return} \PY{n}{result}
\end{Verbatim}
\end{code}


\subsubsection{\texorpdfstring{\textbf{Test}}{Test}}\label{test}

   \begin{code}\begin{Verbatim}[commandchars=\\\{\}]
{\color{incolor}In [{\color{incolor}237}]:} \PY{n}{msg} \PY{o}{=} \PY{l+s+s2}{\PYZdq{}}\PY{l+s+s2}{à Lorem ipsum dolor sit amet, consectetur adipiscing elit. Quisque egestas eget orci}
                 \PY{l+s+s2}{sit amet ullamcorper. Curabitur venenatis non nulla eu ullamcorper. Sed dignissim,}
                 \PY{l+s+s2}{......}
                 \PY{l+s+s2}{tristique at. Morbi augue purus, elementum non posuere ac, pulvinar sit amet odio.}
                 \PY{l+s+s2}{Nulla vitae. }\PY{l+s+s2}{\PYZdq{}}
                 
          \PY{n}{msgElGamal} \PY{o}{=} \PY{n}{messageToElGamal}\PY{p}{(}\PY{n}{msg}\PY{p}{,} \PY{p}{(}\PY{n}{p}\PY{p}{,} \PY{n}{m}\PY{p}{,} \PY{n}{n}\PY{p}{)}\PY{p}{)}
          \PY{n+nb}{print}\PY{p}{(}\PY{l+s+s2}{\PYZdq{}}\PY{l+s+s2}{Le message }\PY{l+s+se}{\PYZbs{}\PYZdq{}}\PY{l+s+s2}{\PYZdq{}}\PY{o}{+}\PY{n}{msg}\PY{o}{+}\PY{l+s+s2}{\PYZdq{}}\PY{l+s+se}{\PYZbs{}\PYZdq{}}\PY{l+s+s2}{\PYZdq{}}\PY{p}{)}
          \PY{n+nb}{print}\PY{p}{(}\PY{l+s+s2}{\PYZdq{}}\PY{l+s+s2}{à été chiffré :}\PY{l+s+s2}{\PYZdq{}}\PY{p}{,}\PY{n}{msgElGamal}\PY{p}{)}
\end{Verbatim}
\end{code}

\newpage
   \begin{code}\begin{Verbatim}[commandchars=\\\{\}]
Le message "à Lorem ipsum dolor sit amet, consectetur adipiscing elit. Quisque egestas eget orci sit
            ......
            Morbi augue purus, elementum non posuere ac, pulvinar sit amet odio. Nulla vitae. "

à été chiffré : [(109434280020782806597252569263,
934309489404227287951670803618982807173118985947282268019), 
(1459792442130535502460205342295, 1903418262967411147829421345996228320231855013638778776853),
......
(863087472162997549675753508325, 976857556035481490247804987927615071380108815839369400033),
(1412585201035017209573687487619, 423490578546975523318750149647570)]

    \end{Verbatim}
\end{code}

\subsection{\texorpdfstring{\textbf{Déchiffrement ElGamal d'un
message}}{Déchiffrement ElGamal d'un message}}\label{duxe9chiffrement-elgamal-dun-message}

\subsubsection{\texorpdfstring{\textbf{Définition}}{Définition}}\label{duxe9finition}

   \begin{code}\begin{Verbatim}[commandchars=\\\{\}]
{\color{incolor}In [{\color{incolor}0}]:} \PY{k}{def} \PY{n+nf}{elGamalToMessage}\PY{p}{(}\PY{n}{elGamal}\PY{p}{,} \PY{n}{prive}\PY{p}{)}\PY{p}{:}
        
          \PY{n}{result} \PY{o}{=} \PY{p}{[}\PY{p}{]}
          \PY{k}{for} \PY{n}{elt} \PY{o+ow}{in} \PY{n}{elGamal}\PY{p}{:}
            \PY{n}{result}\PY{o}{.}\PY{n}{append}\PY{p}{(}\PY{n}{dechiffrementElGamal}\PY{p}{(}\PY{n}{elt}\PY{p}{,} \PY{n}{prive}\PY{p}{)}\PY{p}{)}
          
          \PY{n}{p}\PY{p}{,} \PY{n}{\PYZus{}} \PY{o}{=} \PY{n}{prive}
          \PY{n}{taille} \PY{o}{=} \PY{n}{tailleBlocMax}\PY{p}{(}\PY{n}{p}\PY{p}{)}
          \PY{n}{l} \PY{o}{=} \PY{n}{decodage}\PY{p}{(}\PY{n}{result}\PY{p}{,} \PY{n}{tailleBloc}\PY{o}{=}\PY{n}{taille}\PY{p}{)}
        
          \PY{k}{return} \PY{n}{l}
\end{Verbatim}
\end{code}


\subsubsection{\texorpdfstring{\textbf{Test}}{Test}}\label{test}

Le message est donc bien déchiffré, excepté, encore une fois, pour le
caractère spécial "à" qui a été remplacé par un "@".

   \begin{code}\begin{Verbatim}[commandchars=\\\{\}]
{\color{incolor}In [{\color{incolor}239}]:} \PY{n+nb}{print}\PY{p}{(}\PY{l+s+s2}{\PYZdq{}}\PY{l+s+s2}{Le message à déchiffrer est :}\PY{l+s+s2}{\PYZdq{}}\PY{p}{,}\PY{n}{msgElGamal}\PY{p}{)}
          \PY{n+nb}{print}\PY{p}{(}\PY{l+s+s2}{\PYZdq{}}\PY{l+s+s2}{\PYZdq{}}\PY{p}{)}
          
          \PY{n}{msgAfter} \PY{o}{=} \PY{n}{elGamalToMessage}\PY{p}{(}\PY{n}{msgElGamal}\PY{p}{,} \PY{p}{(}\PY{n}{p}\PY{p}{,} \PY{n}{a}\PY{p}{)}\PY{p}{)}
          \PY{n+nb}{print}\PY{p}{(}\PY{l+s+s2}{\PYZdq{}}\PY{l+s+s2}{Message déchiffré :}\PY{l+s+s2}{\PYZdq{}}\PY{p}{,}\PY{n}{msgAfter}\PY{p}{)}
          \PY{n+nb}{print}\PY{p}{(}\PY{l+s+s2}{\PYZdq{}}\PY{l+s+s2}{     Vérification :}\PY{l+s+s2}{\PYZdq{}}\PY{p}{,}\PY{n}{msg}\PY{p}{)}
\end{Verbatim}
\end{code}


   \begin{code}\begin{Verbatim}[commandchars=\\\{\}]
Le message à déchiffrer est : [(109434280020782806597252569263,
934309489404227287951670803618982807173118985947282268019), 
(1459792442130535502460205342295, 1903418262967411147829421345996228320231855013638778776853),
......
(863087472162997549675753508325, 976857556035481490247804987927615071380108815839369400033),
(1412585201035017209573687487619, 423490578546975523318750149647570)]

Message déchiffré : @ Lorem ipsum dolor sit amet, consectetur adipiscing elit. Quisque egestas eget
                    ......
                    Morbi augue purus, elementum non posuere ac, pulvinar sit amet odio. Nulla vitae.

     Vérification : à Lorem ipsum dolor sit amet, consectetur adipiscing elit. Quisque egestas eget
                    ......
                    Morbi augue purus, elementum non posuere ac, pulvinar sit amet odio. Nulla vitae.

    \end{Verbatim}
\end{code}

\chapter{\texorpdfstring{\textbf{Partie 6 - Arithmétique dans K{[}X{]}
et application à
AES}}{Partie 6 - Arithmétique dans K{[}X{]} et application à AES}}\label{partie-6---arithmuxe9tique-dans-kx-et-application-uxe0-aes}

\section{\texorpdfstring{\textbf{Question
1}}{Question 1}}\label{question-1}

\subsection{\texorpdfstring{\textbf{Implémentation de K{[}X{]} dans un
corps}}{Implémentation de K{[}X{]} dans un corps}}\label{impluxe9mentation-de-kx-dans-un-corps}

Implémentation de la fonction \(inverseMod(x, n)\) pour \(K[X]\) est
demmandé Question 2 mais est requis pour l'implémentation de la divison
de \(PolynomeModRing\)

\subsubsection{\texorpdfstring{\textbf{Définition}}{Définition}}\label{duxe9finition}

Cette implémentation concerne les polynômes à une variable, par exemple
: \(2x^3+4x^2+5x+10\).
Les polynomes sont à coéfficients dans \(F_n[X]\). Ces coefficients sond
donc des entiers \(a_k\) tel que 0 \(\leq\) \(a_k\) \(\lt\) n.

   \begin{code}\begin{Verbatim}[commandchars=\\\{\}]
{\color{incolor}In [{\color{incolor}288}]:} \PY{k}{class} \PY{n+nc}{PolynomeModRing}\PY{p}{:}
          
            \PY{c+c1}{\PYZsh{}Constructor}
            \PY{k}{def} \PY{n+nf}{\PYZus{}\PYZus{}init\PYZus{}\PYZus{}}\PY{p}{(}\PY{n+nb+bp}{self}\PY{p}{,} \PY{n}{p}\PY{p}{,} \PY{n}{n}\PY{p}{)}\PY{p}{:}
              \PY{k}{assert} \PY{n+nb}{isinstance}\PY{p}{(}\PY{n}{p}\PY{p}{,} \PY{n+nb}{list}\PY{p}{)}\PY{p}{,} \PY{l+s+s2}{\PYZdq{}}\PY{l+s+s2}{p doit être sous forme de liste de coefficients}\PY{l+s+s2}{\PYZdq{}}
              \PY{k}{assert} \PY{n+nb}{isinstance}\PY{p}{(}\PY{n}{n}\PY{p}{,} \PY{n+nb}{int}\PY{p}{)}\PY{p}{,} \PY{l+s+s2}{\PYZdq{}}\PY{l+s+s2}{n doit être un entier}\PY{l+s+s2}{\PYZdq{}}
          
              \PY{n+nb+bp}{self}\PY{o}{.}\PY{n}{p} \PY{o}{=} \PY{p}{[}\PY{n}{p}\PY{p}{[}\PY{n}{i}\PY{p}{]}\PY{o}{\PYZpc{}}\PY{k}{n} for i in range(len(p))]
              \PY{n+nb+bp}{self}\PY{o}{.}\PY{n}{n} \PY{o}{=} \PY{n}{n}
            
            \PY{n+nb+bp}{......}
            \PY{n+nb+bp}{......} VOIR DANS LE FICHIER "PolyModRing.py" pour le code complet 
            \PY{n+nb+bp}{......} (cela surchargerai le compte rendu de le mettre en entier ici)
            \PY{n+nb+bp}{......}
            
            \PY{k}{def} \PY{n+nf}{\PYZus{}\PYZus{}str\PYZus{}\PYZus{}}\PY{p}{(}\PY{n+nb+bp}{self}\PY{p}{)}\PY{p}{:}
              \PY{k}{return} \PY{n+nb+bp}{self}\PY{o}{.}\PY{n}{toString}\PY{p}{(}\PY{p}{)}
          
            \PY{k}{def} \PY{n+nf}{toString}\PY{p}{(}\PY{n+nb+bp}{self}\PY{p}{)}\PY{p}{:}
              \PY{c+c1}{\PYZsh{}return str(self.p[len(p)\PYZhy{}1])}
          
              \PY{k}{if} \PY{n+nb+bp}{self}\PY{o}{.}\PY{n}{powerMax}\PY{p}{(}\PY{p}{)} \PY{o}{==} \PY{l+m+mi}{0}\PY{p}{:}
                \PY{k}{return} \PY{l+s+s2}{\PYZdq{}}\PY{l+s+s2}{(}\PY{l+s+s2}{\PYZdq{}}\PY{o}{+}\PY{n+nb}{str}\PY{p}{(}\PY{n+nb+bp}{self}\PY{o}{.}\PY{n}{p}\PY{p}{[}\PY{n+nb+bp}{self}\PY{o}{.}\PY{n}{size}\PY{p}{(}\PY{p}{)}\PY{o}{\PYZhy{}}\PY{l+m+mi}{1}\PY{p}{]}\PY{p}{)}\PY{o}{+}\PY{l+s+s2}{\PYZdq{}}\PY{l+s+s2}{)}\PY{l+s+s2}{\PYZdq{}}
          
              \PY{n}{m} \PY{o}{=} \PY{p}{[}\PY{p}{]}
              \PY{k}{for} \PY{n}{k} \PY{o+ow}{in} \PY{n+nb}{range}\PY{p}{(}\PY{n+nb+bp}{self}\PY{o}{.}\PY{n}{size}\PY{p}{(}\PY{p}{)}\PY{p}{)}\PY{p}{:}
                \PY{k}{if} \PY{p}{(}\PY{n+nb+bp}{self}\PY{o}{.}\PY{n}{p}\PY{p}{[}\PY{n}{k}\PY{p}{]} \PY{o}{!=} \PY{l+m+mi}{0}\PY{p}{)}\PY{p}{:}
                  \PY{n}{m}\PY{o}{.}\PY{n}{append}\PY{p}{(}\PY{p}{(}\PY{n+nb+bp}{self}\PY{o}{.}\PY{n}{p}\PY{p}{[}\PY{n}{k}\PY{p}{]}\PY{p}{,}\PY{n+nb+bp}{self}\PY{o}{.}\PY{n}{size}\PY{p}{(}\PY{p}{)}\PY{o}{\PYZhy{}}\PY{l+m+mi}{1}\PY{o}{\PYZhy{}}\PY{n}{k}\PY{p}{)}\PY{p}{)}
              
              \PY{k}{return} \PY{l+s+s2}{\PYZdq{}}\PY{l+s+s2}{(}\PY{l+s+s2}{\PYZdq{}}\PY{o}{+}\PY{l+s+s2}{\PYZdq{}}\PY{l+s+s2}{ + }\PY{l+s+s2}{\PYZdq{}}\PY{o}{.}\PY{n}{join}\PY{p}{(}\PY{p}{[}\PY{n+nb}{str}\PY{p}{(}\PY{n}{m}\PY{p}{[}\PY{n}{i}\PY{p}{]}\PY{p}{[}\PY{l+m+mi}{0}\PY{p}{]}\PY{p}{)}\PY{o}{+}\PY{l+s+s2}{\PYZdq{}}\PY{l+s+s2}{x\PYZca{}}\PY{l+s+s2}{\PYZdq{}}\PY{o}{+}\PY{n+nb}{str}\PY{p}{(}\PY{n}{m}\PY{p}{[}\PY{n}{i}\PY{p}{]}\PY{p}{[}\PY{l+m+mi}{1}\PY{p}{]}\PY{p}{)} \PY{k}{for} \PY{n}{i} \PY{o+ow}{in} \PY{n+nb}{range}\PY{p}{(}\PY{n+nb}{len}\PY{p}{(}\PY{n}{m}\PY{p}{)}\PY{p}{)}\PY{p}{]}\PY{p}{)}\PY{o}{+}\PY{l+s+s2}{\PYZdq{}}\PY{l+s+s2}{)}\PY{l+s+s2}{\PYZdq{}}
          
\end{Verbatim}
\end{code}

\subsubsection{\texorpdfstring{\textbf{Test}}{Test}}\label{test}

   \begin{code}\begin{Verbatim}[commandchars=\\\{\}]
{\color{incolor}In [{\color{incolor}289}]:} \PY{n}{n} \PY{o}{=} \PY{l+m+mi}{29}
          \PY{n}{p} \PY{o}{=} \PY{p}{[}\PY{l+m+mi}{1}\PY{p}{,} \PY{l+m+mi}{0}\PY{p}{,} \PY{l+m+mi}{0}\PY{p}{,} \PY{l+m+mi}{0}\PY{p}{,} \PY{l+m+mi}{0}\PY{p}{,} \PY{l+m+mi}{0}\PY{p}{,} \PY{l+m+mi}{0}\PY{p}{,} \PY{l+m+mi}{0}\PY{p}{,} \PY{l+m+mi}{0}\PY{p}{,} \PY{l+m+mi}{0}\PY{p}{,} \PY{l+m+mi}{15}\PY{p}{]}
          \PY{n}{q} \PY{o}{=} \PY{p}{[}\PY{o}{\PYZhy{}}\PY{l+m+mi}{1}\PY{p}{,} \PY{l+m+mi}{0}\PY{p}{,} \PY{l+m+mi}{15}\PY{p}{]}
          
          \PY{n}{P} \PY{o}{=} \PY{n}{PolynomeModRing}\PY{p}{(}\PY{n}{p}\PY{p}{,} \PY{n}{n}\PY{p}{)}
          \PY{n}{Q} \PY{o}{=} \PY{n}{PolynomeModRing}\PY{p}{(}\PY{n}{q}\PY{p}{,} \PY{n}{n}\PY{p}{)}
          
          \PY{n}{R} \PY{o}{=} \PY{n}{P}\PY{o}{+}\PY{n}{Q}
          \PY{n+nb}{print}\PY{p}{(}\PY{l+s+s2}{\PYZdq{}}\PY{l+s+s2}{Addition de deux polynômes P+Q :}\PY{l+s+s2}{\PYZdq{}}\PY{p}{,} \PY{n}{P}\PY{p}{,}\PY{l+s+s2}{\PYZdq{}}\PY{l+s+s2}{+}\PY{l+s+s2}{\PYZdq{}}\PY{p}{,} \PY{n}{Q}\PY{p}{,}\PY{l+s+s2}{\PYZdq{}}\PY{l+s+se}{\PYZbs{}n}\PY{l+s+s2}{=}\PY{l+s+s2}{\PYZdq{}}\PY{p}{,} \PY{n}{R}\PY{p}{)}
          \PY{n+nb}{print}\PY{p}{(}\PY{l+s+s2}{\PYZdq{}}\PY{l+s+s2}{\PYZdq{}}\PY{p}{)}
          
          \PY{n}{k} \PY{o}{=} \PY{l+m+mi}{2}
          \PY{n}{R} \PY{o}{=} \PY{n}{P}\PY{o}{*}\PY{n}{k}
          \PY{n+nb}{print}\PY{p}{(}\PY{l+s+s2}{\PYZdq{}}\PY{l+s+s2}{Produit d}\PY{l+s+s2}{\PYZsq{}}\PY{l+s+s2}{un polynôme et d}\PY{l+s+s2}{\PYZsq{}}\PY{l+s+s2}{un entier P*2 :}\PY{l+s+s2}{\PYZdq{}}\PY{p}{,}\PY{n}{P}\PY{p}{,}\PY{l+s+s2}{\PYZdq{}}\PY{l+s+s2}{*}\PY{l+s+s2}{\PYZdq{}}\PY{p}{,}\PY{n}{k}\PY{p}{,}\PY{l+s+s2}{\PYZdq{}}\PY{l+s+se}{\PYZbs{}n}\PY{l+s+s2}{=}\PY{l+s+s2}{\PYZdq{}}\PY{p}{,}\PY{n}{R}\PY{p}{)}
          \PY{n+nb}{print}\PY{p}{(}\PY{l+s+s2}{\PYZdq{}}\PY{l+s+s2}{\PYZdq{}}\PY{p}{)}
          
          \PY{n}{R} \PY{o}{=} \PY{n}{P}\PY{o}{*}\PY{n}{Q}
          \PY{n+nb}{print}\PY{p}{(}\PY{l+s+s2}{\PYZdq{}}\PY{l+s+s2}{Produit de deux polynômes P*Q :}\PY{l+s+s2}{\PYZdq{}}\PY{p}{,} \PY{n}{P}\PY{p}{,} \PY{l+s+s2}{\PYZdq{}}\PY{l+s+s2}{*}\PY{l+s+s2}{\PYZdq{}}\PY{p}{,} \PY{n}{Q}\PY{p}{,} \PY{l+s+s2}{\PYZdq{}}\PY{l+s+se}{\PYZbs{}n}\PY{l+s+s2}{=}\PY{l+s+s2}{\PYZdq{}}\PY{p}{,}\PY{n}{R}\PY{p}{)}
          \PY{n+nb}{print}\PY{p}{(}\PY{l+s+s2}{\PYZdq{}}\PY{l+s+s2}{\PYZdq{}}\PY{p}{)}
          
          \PY{n}{R} \PY{o}{=} \PY{n}{P}\PY{o}{/}\PY{o}{/}\PY{n}{Q}
          \PY{n+nb}{print}\PY{p}{(}\PY{l+s+s2}{\PYZdq{}}\PY{l+s+s2}{Division euclidienne de deux polynômes P//Q=R :}\PY{l+s+s2}{\PYZdq{}}\PY{p}{,} \PY{n}{P}\PY{p}{,} \PY{l+s+s2}{\PYZdq{}}\PY{l+s+s2}{//}\PY{l+s+s2}{\PYZdq{}}\PY{p}{,} \PY{n}{Q}\PY{p}{,} \PY{l+s+s2}{\PYZdq{}}\PY{l+s+se}{\PYZbs{}n}\PY{l+s+s2}{=}\PY{l+s+s2}{\PYZdq{}}\PY{p}{,}\PY{n}{R}\PY{p}{)}
\end{Verbatim}
\end{code}


   \begin{code}\begin{Verbatim}[commandchars=\\\{\}]
Addition de deux polynômes P+Q : (1x\^{}10 + 15x\^{}0) + (28x\^{}2 + 15x\^{}0) 
= (1x\^{}10 + 28x\^{}2 + 1x\^{}0)

Produit d'un polynôme et d'un entier P*2 : (1x\^{}10 + 15x\^{}0) * 2 
= (2x\^{}10 + 1x\^{}0)

Produit de deux polynômes P*Q : (1x\^{}10 + 15x\^{}0) * (28x\^{}2 + 15x\^{}0) 
= (28x\^{}12 + 15x\^{}10 + 14x\^{}2 + 22x\^{}0)

Division euclidienne de deux polynômes P//Q=R : (1x\^{}10 + 15x\^{}0) // (28x\^{}2 + 15x\^{}0) 
= (28x\^{}8 + 14x\^{}6 + 7x\^{}4 + 18x\^{}2 + 9x\^{}0)

    \end{Verbatim}
\end{code}

\section{\texorpdfstring{\textbf{Question
2}}{Question 2}}\label{question-2}

\subsection{\texorpdfstring{\textbf{Euclide Etendu dans
\(F_n[X]\)}}{Euclide Etendu dans F\_n{[}X{]}}}\label{euclide-etendu-dans-f_nx}

\subsubsection{\texorpdfstring{\textbf{Re-Définition}}{Re-Définition}}\label{re-duxe9finition}

   \begin{code}\begin{Verbatim}[commandchars=\\\{\}]
{\color{incolor}In [{\color{incolor}0}]:} \PY{k}{def} \PY{n+nf}{euclideEtendu}\PY{p}{(}\PY{n}{a}\PY{p}{,} \PY{n}{b}\PY{p}{)}\PY{p}{:}
          
          \PY{n}{r}\PY{p}{,} \PY{n}{r0} \PY{o}{=} \PY{n}{a}\PY{p}{,} \PY{n}{b}
          \PY{n}{u}\PY{p}{,} \PY{n}{v}\PY{p}{,} \PY{n}{u0}\PY{p}{,} \PY{n}{v0} \PY{o}{=} \PY{k+kc}{None}\PY{p}{,} \PY{k+kc}{None}\PY{p}{,} \PY{k+kc}{None}\PY{p}{,} \PY{k+kc}{None}
        
          \PY{k}{if} \PY{n+nb}{isinstance}\PY{p}{(}\PY{n}{a}\PY{p}{,} \PY{n+nb}{int}\PY{p}{)}\PY{p}{:}
            \PY{n}{u}\PY{p}{,} \PY{n}{v}\PY{p}{,} \PY{n}{u0}\PY{p}{,} \PY{n}{v0} \PY{o}{=} \PY{l+m+mi}{1}\PY{p}{,} \PY{l+m+mi}{0}\PY{p}{,} \PY{l+m+mi}{0}\PY{p}{,} \PY{l+m+mi}{1}
          \PY{k}{elif} \PY{n+nb}{isinstance}\PY{p}{(}\PY{n}{a}\PY{p}{,} \PY{n}{PolynomeModRing}\PY{p}{)}\PY{p}{:}
            \PY{n}{u}\PY{p}{,} \PY{n}{v}\PY{p}{,} \PY{n}{u0}\PY{p}{,} \PY{n}{v0} \PY{o}{=} \PY{n}{PolynomeModRing}\PY{p}{(}\PY{p}{[}\PY{l+m+mi}{1}\PY{p}{]}\PY{p}{,} \PY{n}{a}\PY{o}{.}\PY{n}{n}\PY{p}{)}\PY{p}{,} \PY{n}{PolynomeModRing}\PY{p}{(}\PY{p}{[}\PY{l+m+mi}{0}\PY{p}{]}\PY{p}{,} \PY{n}{a}\PY{o}{.}\PY{n}{n}\PY{p}{)}\PY{p}{,}
                           \PY{n}{PolynomeModRing}\PY{p}{(}\PY{p}{[}\PY{l+m+mi}{0}\PY{p}{]}\PY{p}{,} \PY{n}{a}\PY{o}{.}\PY{n}{n}\PY{p}{)}\PY{p}{,} \PY{n}{PolynomeModRing}\PY{p}{(}\PY{p}{[}\PY{l+m+mi}{1}\PY{p}{]}\PY{p}{,} \PY{n}{a}\PY{o}{.}\PY{n}{n}\PY{p}{)}
          \PY{k}{else}\PY{p}{:}
            \PY{k}{raise} \PY{n+ne}{Exception}\PY{p}{(}\PY{l+s+s2}{\PYZdq{}}\PY{l+s+s2}{TypeError}\PY{l+s+s2}{\PYZdq{}}\PY{p}{,}\PY{l+s+s2}{\PYZdq{}}\PY{l+s+s2}{Cette fonction a été définie pour des paramètres de type}
                                         \PY{l+s+s2}{entiers ou polynomes}\PY{l+s+s2}{\PYZdq{}}\PY{p}{)}
        
          \PY{k}{while} \PY{p}{(}\PY{n+nb}{isinstance}\PY{p}{(}\PY{n}{r0}\PY{p}{,} \PY{n+nb}{int}\PY{p}{)} \PY{o+ow}{and} \PY{n}{r0}\PY{p}{)} \PY{o+ow}{or} \PY{p}{(}\PY{n+nb}{isinstance}\PY{p}{(}\PY{n}{r0}\PY{p}{,} \PY{n}{PolynomeModRing}\PY{p}{)} \PY{o+ow}{and}
                \PY{p}{(}\PY{n}{r0}\PY{o}{.}\PY{n}{powerMax}\PY{p}{(}\PY{p}{)}\PY{o}{!=}\PY{l+m+mi}{0} \PY{o+ow}{or} \PY{n}{r0}\PY{o}{.}\PY{n}{p}\PY{p}{[}\PY{n}{r0}\PY{o}{.}\PY{n}{size}\PY{p}{(}\PY{p}{)}\PY{o}{\PYZhy{}}\PY{l+m+mi}{1}\PY{p}{]}\PY{p}{)}\PY{p}{)}\PY{p}{:}
            \PY{n}{q} \PY{o}{=} \PY{n}{r}\PY{o}{/}\PY{o}{/}\PY{n}{r0}
            \PY{n}{r}\PY{p}{,} \PY{n}{u}\PY{p}{,} \PY{n}{v}\PY{p}{,} \PY{n}{r0}\PY{p}{,} \PY{n}{u0}\PY{p}{,} \PY{n}{v0} \PY{o}{=} \PY{n}{r0}\PY{p}{,} \PY{n}{u0}\PY{p}{,} \PY{n}{v0}\PY{p}{,} \PY{n}{r}\PY{o}{\PYZhy{}}\PY{p}{(}\PY{n}{q}\PY{o}{*}\PY{n}{r0}\PY{p}{)}\PY{p}{,} \PY{n}{u}\PY{o}{\PYZhy{}}\PY{p}{(}\PY{n}{q}\PY{o}{*}\PY{n}{u0}\PY{p}{)}\PY{p}{,} \PY{n}{v}\PY{o}{\PYZhy{}}\PY{p}{(}\PY{n}{q}\PY{o}{*}\PY{n}{v0}\PY{p}{)}
        
          \PY{k}{return} \PY{n}{r}\PY{p}{,} \PY{n}{u}\PY{p}{,} \PY{n}{v}
\end{Verbatim}
\end{code}


\subsubsection{\texorpdfstring{\textbf{Test}}{Test}}\label{test}

   \begin{code}\begin{Verbatim}[commandchars=\\\{\}]
{\color{incolor}In [{\color{incolor}292}]:} \PY{n}{R}\PY{p}{,} \PY{n}{U}\PY{p}{,} \PY{n}{V} \PY{o}{=} \PY{n}{euclideEtendu}\PY{p}{(}\PY{n}{P}\PY{p}{,} \PY{n}{Q}\PY{p}{)}
          
          \PY{n+nb}{print}\PY{p}{(}\PY{l+s+s2}{\PYZdq{}}\PY{l+s+s2}{Le PGCD de}\PY{l+s+s2}{\PYZdq{}}\PY{p}{,} \PY{n}{P}\PY{p}{,} \PY{l+s+s2}{\PYZdq{}}\PY{l+s+s2}{et de}\PY{l+s+s2}{\PYZdq{}}\PY{p}{,} \PY{n}{Q}\PY{p}{,} \PY{l+s+s2}{\PYZdq{}}\PY{l+s+s2}{est :}\PY{l+s+s2}{\PYZdq{}}\PY{p}{,}\PY{n}{R}\PY{p}{)}
          \PY{n+nb}{print}\PY{p}{(}\PY{l+s+s2}{\PYZdq{}}\PY{l+s+s2}{\PYZdq{}}\PY{p}{)}
          \PY{n+nb}{print}\PY{p}{(}\PY{l+s+s2}{\PYZdq{}}\PY{l+s+s2}{De plus : R=P*U+Q*V avec U =}\PY{l+s+s2}{\PYZdq{}}\PY{p}{,}\PY{n}{U}\PY{p}{,}\PY{l+s+s2}{\PYZdq{}}\PY{l+s+s2}{ et V =}\PY{l+s+s2}{\PYZdq{}}\PY{p}{,}\PY{n}{V}\PY{p}{)}
          \PY{n+nb}{print}\PY{p}{(}\PY{l+s+s2}{\PYZdq{}}\PY{l+s+s2}{\PYZdq{}}\PY{p}{)}
          \PY{n+nb}{print}\PY{p}{(}\PY{l+s+s2}{\PYZdq{}}\PY{l+s+s2}{Verification :}\PY{l+s+s2}{\PYZdq{}}\PY{p}{,}\PY{n}{P}\PY{p}{,}\PY{l+s+s2}{\PYZdq{}}\PY{l+s+s2}{*}\PY{l+s+s2}{\PYZdq{}}\PY{p}{,}\PY{n}{U}\PY{p}{,}\PY{l+s+s2}{\PYZdq{}}\PY{l+s+s2}{+}\PY{l+s+s2}{\PYZdq{}}\PY{p}{,}\PY{n}{Q}\PY{p}{,}\PY{l+s+s2}{\PYZdq{}}\PY{l+s+s2}{*}\PY{l+s+s2}{\PYZdq{}}\PY{p}{,}\PY{n}{V}\PY{p}{,}\PY{l+s+s2}{\PYZdq{}}\PY{l+s+se}{\PYZbs{}n}\PY{l+s+s2}{=}\PY{l+s+s2}{\PYZdq{}}\PY{p}{,}\PY{n}{P}\PY{o}{*}\PY{n}{U}\PY{o}{+}\PY{n}{Q}\PY{o}{*}\PY{n}{V}\PY{p}{)}
\end{Verbatim}
\end{code}


   \begin{code}\begin{Verbatim}[commandchars=\\\{\}]
[22, 0, 18]
Le PGCD de (1x\^{}10 + 15x\^{}0) et de (28x\^{}2 + 15x\^{}0) est : (25)

De plus : R=P*U+Q*V avec U = (1)  et V = (1x\^{}8 + 15x\^{}6 + 22x\^{}4 + 11x\^{}2 + 20x\^{}0)

Verification : (1x\^{}10 + 15x\^{}0) * (1) + (28x\^{}2 + 15x\^{}0) * (1x\^{}8 + 15x\^{}6 + 22x\^{}4 + 11x\^{}2 + 20x\^{}0) 
= (25)

    \end{Verbatim}
\end{code}

\subsection{\texorpdfstring{\textbf{Inverse Modulaire dans
\(F_n[X]\)}}{Inverse Modulaire dans F\_n{[}X{]}}}\label{inverse-modulaire-dans-f_nx}

\subsubsection{\texorpdfstring{\textbf{Re-Définition}}{Re-Définition}}\label{duxe9finition}

   \begin{code}\begin{Verbatim}[commandchars=\\\{\}]
{\color{incolor}In [{\color{incolor}0}]:} \PY{k}{def} \PY{n+nf}{inverseMod}\PY{p}{(}\PY{n}{x}\PY{p}{,} \PY{n}{n}\PY{p}{)}\PY{p}{:}
          \PY{n}{r}\PY{p}{,} \PY{n}{u}\PY{p}{,} \PY{n}{\PYZus{}} \PY{o}{=} \PY{n}{euclideEtendu}\PY{p}{(}\PY{n}{x}\PY{p}{,} \PY{n}{n}\PY{p}{)}
          
          \PY{k}{if} \PY{n+nb}{isinstance}\PY{p}{(}\PY{n}{r}\PY{p}{,} \PY{n+nb}{int}\PY{p}{)}\PY{p}{:}
            \PY{k}{assert} \PY{n}{r}\PY{o}{==}\PY{l+m+mi}{1}\PY{p}{,} \PY{l+s+s2}{\PYZdq{}}\PY{l+s+s2}{x et n doivent être premiers entre eux}\PY{l+s+s2}{\PYZdq{}}
        
          \PY{k}{if} \PY{n+nb}{isinstance}\PY{p}{(}\PY{n}{r}\PY{p}{,} \PY{n}{PolynomeModRing}\PY{p}{)}\PY{p}{:}
            \PY{k}{assert} \PY{n}{r}\PY{o}{.}\PY{n}{powerMax}\PY{p}{(}\PY{p}{)} \PY{o+ow}{and} \PY{n}{r}\PY{o}{.}\PY{n}{p}\PY{p}{[}\PY{n}{r}\PY{o}{.}\PY{n}{size}\PY{p}{(}\PY{p}{)}\PY{o}{\PYZhy{}}\PY{l+m+mi}{1}\PY{p}{]}\PY{o}{==}\PY{p}{[}\PY{l+m+mi}{1}\PY{p}{]}\PY{p}{,} \PY{l+s+s2}{\PYZdq{}}\PY{l+s+s2}{x et n doivent être premiers entre eux}\PY{l+s+s2}{\PYZdq{}}
        
          \PY{k}{return} \PY{n}{u}\PY{o}{\PYZpc{}}\PY{k}{n}
\end{Verbatim}
\end{code}


\subsubsection{\texorpdfstring{\textbf{Test}}{Test}}\label{test}

   \begin{code}\begin{Verbatim}[commandchars=\\\{\}]
{\color{incolor}In [{\color{incolor}0}]:} \PY{c+c1}{\PYZsh{}NE FONCTIONNE PAS}
        
        \PY{c+c1}{\PYZsh{}N = PolynomeModRing([n], n+1)}
        
        \PY{c+c1}{\PYZsh{}R = inverseMod(P, N)}
        \PY{c+c1}{\PYZsh{}print(\PYZdq{}L\PYZsq{}inverse de\PYZdq{},P,\PYZdq{}modulo\PYZdq{},n,\PYZdq{}est :\PYZdq{},R)}
\end{Verbatim}
\end{code}


\section{\texorpdfstring{\textbf{Question
2}}{Question 2}}\label{question-2}

\subsection{\texorpdfstring{\textbf{Utilisation de cette implémentation
dans la routine SubBytes
d'AES}}{Utilisation de cette implémentation dans la routine SubBytes d'AES}}\label{utilisation-de-cette-impluxe9mentation-dans-la-routine-subbytes-daes}

Cette partie du chiffrement AES consiste à effectuer des clalculs sur
des bits sous forme de polynôme, il faudra donc utiliser pour chaque
octet \(PolynomeModRing([a_7,a_6,a_5,a_4,a_3,a_2,a_1, a_0], 2)\).
Ensuite il sera donc possible d'effectuer les opérations sur ces
polynômes afin d'obtenir la matrice de transformation afin de
l'application sur chaque octet de l'entrée.


    % Add a bibliography block to the postdoc
    
    
    
    \end{document}
